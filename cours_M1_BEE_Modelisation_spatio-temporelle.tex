\documentclass[]{beamer}
\usepackage[utf8]{inputenc}
\usepackage[T1]{fontenc}
\usepackage[english]{babel}
\usepackage{natbib}
\usepackage{hyperref}
% \usepackage{animate}
\usepackage{graphicx}
\usepackage{color}

%% Theme
\mode<presentation>
% \usetheme[]{Berlin}
\useoutertheme[footline=authorinstitutetitle]{miniframes}
\useoutertheme{smoothtree}
\useinnertheme[shadow=true]{rounded}
\usecolortheme{orchid}
\usecolortheme{whale}


\graphicspath{{./images/}} 


% \usecolortheme{progressbar}
% \usefonttheme{progressbar}
% \useinnertheme{progressbar}
% \useoutertheme{progressbar}
%% Logo
% \logo{\includegraphics[height=0.5cm]{Logo.png}}
%% Fond de la page
% \setbeamercolor{background canvas}{bg=couleur}

\title[Dyna: Dynamiques spatiales]{Les dynamiques spatio-temporelles des communautés et des populations.\\
  \textit{\footnotesize{Plusieurs exemples de modélisation}}.}

\author{Romain Lorrillière | \textit{romain.lorrilliere@mnhn.fr}}
\institute{Paris Saclay : M1 BEE Module DYNAMIQUE ET GESTION DES POPULATIONS (Fev 2022)}
\date{8 Février 2022}

\AtBeginSection[]{
  \begin{frame}
    \frametitle{\insertsectionhead}
    \scriptsize \tableofcontents[currentsection,hideothersubsections]
  \end{frame}
}

\addtobeamertemplate{footline}{\insertframenumber/\inserttotalframenumber}

\begin{document}
\maketitle



% \section{Sommaire}
%% =====================
% \begin{frame}
%   \frametitle{Sommaire}
%   \tableofcontents[hideallsubsections]
% \end{frame}


\begin{frame}{Plan}
  \tableofcontents[pausesections]
\end{frame}




% =======================================================
\section{Introduction}
% ===============================================

% ------------------------------------------------
\subsection{La problématique}
% --------------------------------------

\begin{frame}
  \frametitle{L écologie définition}
  \begin{columns}
    \begin{column}[c]{0.5\textwidth}
      \begin{block}{Écologie}
        Compréhension des relations entre les espèces et leur environnement 
      \end{block}
      \begin{block}<5>{Écologie du paysage}
        Prise en compte de l’Hétérogénéité, de l’homme et des différentes échelles
      \end{block}
    \end{column}
    \begin{column}[l]{0.5\textwidth}
      \includegraphics<1>[width=\textwidth]{niche1}
      \includegraphics<2>[width=\textwidth]{niche2}
      \includegraphics<3>[width=\textwidth]{niche3}    
      \includegraphics<4>[width=\textwidth]{marnage}   
      \includegraphics<5>[width=\textwidth]{ecologiePaysage_idiana}   
    \end{column}
  \end{columns}
\end{frame}


\begin{frame}
  \frametitle{L'écologie du paysage}
  \begin{columns}
    \begin{column}[c]{0.6\textwidth}
      \begin{itemize}
      \item Caractérisation du rôle de la structure du paysage
        \begin{itemize}
        \item habitat
        \item linéaire (route, rivière)
        \end{itemize}

      \item Utilisation des Système d'Information Géographique (SIG ou
        GIS)
        \begin{itemize}
        \item QGIS
        \item R library(sf)
        \item PostgreSQL Postgis          
        \end{itemize}
      \end{itemize}
    \end{column}
    \begin{column}[l]{0.4\textwidth}
      \includegraphics[width=\textwidth]{sig}    
    \end{column}
  \end{columns}
\end{frame}


\begin{frame}{Les dynamiques spatio-temporelles}
  \begin{center}
    \includegraphics<1>[width=.9\textwidth]{dynSpatial1}
    \includegraphics<2>[width=.9\textwidth]{dynSpatial2}
    \includegraphics<3>[width=.9\textwidth]{dynSpatial3}
    \includegraphics<4>[width=.9\textwidth]{dynSpatial4}
    \includegraphics<5>[width=.9\textwidth]{dynSpatial5} 
  \end{center}
\end{frame}




% ------------------------------------------------
\subsection{Définition des modèles et leurs intérêts}
% ------------------------------------------------

\begin{frame}{Définition}
  \begin{alertblock}{Définition : Modèle}
    \begin{itemize}
    \item Petit Robert (1996): […] 7. Sc. Représentation simplifiée d’un processus, d’un système.
      \vspace{10pt}
    \item Wikipedia : un concept ou objet considéré comme représentatif d’un autre, déjà existant ou que l'on va s'efforcer de construire .\\
      exemple : le « modèle réduit » ou maquette, le « modèle » du scientifique
    \end{itemize}
  \end{alertblock}
\end{frame}

\begin{frame}{Des outils très variés}
  \begin{columns}
    \begin{column}[c]{0.6\textwidth}
      \begin{block}{Mathématique appliquées}
        \begin{footnotesize}
           \begin{itemize}[<+->]          
        \item Dynamique de population
          \begin{itemize}
          \item Si pas de contrainte croissance exponentielle \\~\\
            $N_{t+1} = rN_t$ \\
            où $r$ est le taux de croissance\\~\\
          \item mais croissance pas infinie \\
            $\rightarrow$ densité dépendence négative \\~\\
            $N_{t+1} = rN_t(1-\frac{N_t}{K})$ \\
            où $K$ est la capacité de charge pour l'espèce
          \end{itemize}
        
        \end{itemize}
 
        \end{footnotesize}
             \end{block}
  
    \end{column}
    \begin{column}[c]{0.5\textwidth}
      \begin{center}
          \includegraphics<2>[width=\textwidth]{croissance_exp}
          \includegraphics<3>[width=\textwidth]{croissance_logistique}

      
      \end{center}
    \end{column}
  \end{columns} 
\end{frame}


\begin{frame}{Des outils très variés}
  \begin{columns}
    \begin{column}[c]{0.6\textwidth}
      \begin{block}{Mathématique appliquées}
        \begin{footnotesize}
           \begin{itemize}[<+->]          
        
        \item Dynamique de proie prédateur (Lotka-Volterra)
          \begin{itemize}
          \item $\frac{dH}{dt}=r_HH - b_HHP$
          \item $\frac{dP}{dt}=r_Pb_HHP - b_PP$
          \item où $r$ les coefficient d'accroisement et $b$ les taux de mortalité
          \end{itemize}
        \end{itemize}
 
        \end{footnotesize}
             \end{block}
  
    \end{column}
    \begin{column}[c]{0.5\textwidth}
      \begin{center}
           \includegraphics<1-3>[width=.75\textwidth]{lynx_lievre_dyn}
           \includegraphics<4>[width=\textwidth]{Lotka-Volterra_orbits_02}
           \includegraphics<5->[width=.9\textwidth]{Lotka-Volterra_orbits_01}
      
      \end{center}
    \end{column}
  \end{columns} 
\end{frame}


\begin{frame}{Des outils très variés}
  \begin{columns}
    \begin{column}[c]{0.6\textwidth}
      \begin{block}{Statistiques}
        \begin{footnotesize}
           \begin{itemize}[<+->]          
        \item Recherche de corrélation significative, c'est à dire de
          relation
          \begin{itemize}
          \item sens de la relation
          \item force de la relation
          \end{itemize}
      
 \end{itemize}
        \end{footnotesize}
              \end{block}
  
    \end{column}
    \begin{column}[c]{0.5\textwidth}
      \begin{center}
          \includegraphics<1->[width=\textwidth]{correlation}
       \end{center}
    \end{column}
  \end{columns} 
\end{frame}



\begin{frame}{Des outils très variés}
  \begin{columns}
    \begin{column}[c]{0.6\textwidth}
      \begin{block}{Statistiques}
        \begin{footnotesize}
           \begin{itemize}[<+->]          
                \item Attention à l'effet cigogne
          \begin{itemize}
          \item Corrélation n'est pas causalité
          \item \tiny{https://www.lemonde.fr/les-decodeurs/article/2019/03/01/correlations-ou-causalite-generez-vos-propres-cartes-pour-ne-rien-demontrer-du-tout$\_$5430063$\_$4355770.html}
          \end{itemize}
         
        \end{itemize}

        \end{footnotesize}
              \end{block}
  
    \end{column}
    \begin{column}[c]{0.5\textwidth}
      \begin{center}
          \includegraphics<1>[width=.8\textwidth]{naissance_cigogne}
          \includegraphics<2>[width=.8\textwidth]{correlation1}
        \includegraphics<3->[width=.9\textwidth]{correlation2}
       \end{center}
    \end{column}
  \end{columns} 
\end{frame}





\begin{frame}{Des outils très variés}
  \begin{columns}
    \begin{column}[c]{0.6\textwidth}
      \begin{block}{Automates cellulaires}
        \begin{footnotesize}
               Règles simples de dynamique spatiale dans un espace discret
        \begin{itemize}[<+->]          
        \item Modélisation des incendies de forêt
        \item Jeu de la vie
  
        \end{footnotesize}
                  
        \end{itemize}
      \end{block}
      
    \end{column}
    \begin{column}[c]{0.5\textwidth}
      \begin{center}
        \includegraphics<1>[width=\textwidth]{feu}
        \includegraphics<2>[width=\textwidth]{gosper-1}
        \includegraphics<3>[width=\textwidth]{gosper-2}
        \includegraphics<4>[width=\textwidth]{gosper-3}
        \includegraphics<5>[width=\textwidth]{gosper-4}
        \includegraphics<6>[width=\textwidth]{gosper-5}
        \includegraphics<7>[width=\textwidth]{gosper-6}
        \includegraphics<8>[width=\textwidth]{gosper-7}
        \includegraphics<9>[width=\textwidth]{gosper-8}
        \includegraphics<10>[width=\textwidth]{gosper-9}
        \includegraphics<11>[width=\textwidth]{gosper-10}
        \includegraphics<12>[width=\textwidth]{gosper-11}
        \includegraphics<13>[width=\textwidth]{gosper-12}
        \includegraphics<14>[width=\textwidth]{gosper-13}
        \includegraphics<15>[width=\textwidth]{gosper-14}
        \includegraphics<16>[width=\textwidth]{gosper-15}
        \includegraphics<17>[width=\textwidth]{gosper-16}
        \includegraphics<18>[width=\textwidth]{gosper-17}
        \includegraphics<19>[width=\textwidth]{gosper-18}
        \includegraphics<20>[width=\textwidth]{gosper-19}
        \includegraphics<21>[width=\textwidth]{gosper-20}
      \end{center}
    \end{column}
  \end{columns} 
\end{frame}




\begin{frame}{Des outils très variés}
  \begin{columns}
    \begin{column}[c]{0.6\textwidth}
      \begin{block}{Automates cellulaires}
        \begin{footnotesize}
              Règles simples de dynamique spatiale dans un espace discret
        \begin{itemize}       
        \item Modélisation des incendies de forêt
        \item Jeu de la vie
          \begin{itemize}[<+->]
          \item Combien de règles ?
          \item 2
            \begin{itemize}
            \item une cellule morte possédant exactement trois voisines
              vivantes devient vivante (elle naît) 
            \item une cellule vivante possédant deux ou trois voisines vivantes le reste, sinon elle meurt.
            \end{itemize}
          \end{itemize}
          
        \end{itemize}
  
        \end{footnotesize}
        \end{block}
      
    \end{column}
    \begin{column}[c]{0.5\textwidth}
      \begin{center}
        \includegraphics<1>[width=\textwidth]{gosper-16}
        \includegraphics<2>[width=\textwidth]{gosper-17}
        \includegraphics<3>[width=\textwidth]{gosper-18}
        \includegraphics<4>[width=\textwidth]{gosper-19}
        \includegraphics<5>[width=\textwidth]{gosper-20}
      \end{center}
    \end{column}
  \end{columns} 
\end{frame}





\begin{frame}{Des outils très variés}
  \begin{columns}
    \begin{column}[c]{0.6\textwidth}
      \begin{block}{Modèles individus centrés}
        \begin{footnotesize}
          \begin{itemize}[<+->]       
        \item Modélisation des individus et de la gestion de leur choix de forêt
        \item Peut être très complexe
        \item SMA: Système multi-agent (outil : cormas)
        \end{itemize}
  
        \end{footnotesize}
        \end{block}
      
    \end{column}
    \begin{column}[c]{0.5\textwidth}
      \begin{center}
         \includegraphics<1>[width=\textwidth]{gauchedroite}
         \includegraphics<2>[width=\textwidth]{diagramIBM}
         \includegraphics<3>[width=\textwidth]{sma_ferber_1995}
      \end{center}
    \end{column}
  \end{columns} 
\end{frame}




\begin{frame}{Un modéle n'est qu'un modèle}
\begin{block}{}
	Rien de plus qu'une représentation du réel avec des imperfections et des biais
 	\begin{center}
         \includegraphics[width=0.7\textwidth]{pipe_magritte}
      \end{center}
\end{block}
\end{frame}



\section{Dynamiques de populations}
% ===============================================


% ------------------------------------------------
\subsection{Digression: Les modèles de niche}
% ------------------------------------------------

\begin{frame}{Modélisation de niche actuelle}
  \begin{center}
    \includegraphics[width=\textwidth]{nbArticles1978}
  \end{center}
\end{frame}


\begin{frame}{Modélisation de niche actuelle}
  \begin{center}
    \includegraphics[width=\textwidth]{modelNiche1}
  \end{center}
\end{frame}

\begin{frame}{Modélisation de niche futur}
  \begin{center}
    \includegraphics[width=\textwidth]{modelNiche2}
  \end{center}
\end{frame}


\begin{frame}{Modélisation... les consensus (BIOMOD)}
  \begin{center}
    \includegraphics[width=\textwidth]{modelNicheConsensus}
  \end{center}
\end{frame}

\begin{frame}{Niche thermique}
  \begin{center}
    \includegraphics[width=.9\textwidth]{rangeThermic}
  \end{center}

\end{frame}

\begin{frame}{Niche thermique exemple}
  \begin{columns}[c]
    \begin{column}[c]{0.5\textwidth}
      \begin{center}
        Bruant jaune \textit{Emberiza citrinella}
        \vspace{10pt}
        \includegraphics[width=.6\textwidth]{bruantjaune}
        \vspace{5pt}
        \includegraphics[width=.8\textwidth]{distributionBruantJaune}   
      \end{center}
    \end{column}
    \begin{column}[c]{0.5\textwidth}
      
      \begin{center}
        Bruant zizi \textit{Emberiza cirlus}
        \vspace{10pt}
        \includegraphics[width=.6\textwidth]{bruantzizi}
        \vspace{5pt}
        \includegraphics[width=.8\textwidth]{distributionBruantZizi}   
      \end{center}
    \end{column}
  \end{columns}
\end{frame}

\begin{frame}{Distribution hivernal: Des dynamiques différentes}
  \begin{center}
    \includegraphics[width=.9\textwidth]{lanioFicedula}
  \end{center}
\end{frame}

\begin{frame}{Aire de distribution estival}
  \begin{center}
    \includegraphics[width=.9\textwidth]{nicheRepro}
  \end{center}
\end{frame}


\begin{frame}{Ex: Outarde de Macqueen \textit{Chlamydotis macqueenii}}
  \begin{center}
    \includegraphics[width=.9\textwidth]{mcQueen_male}
  \end{center}
\end{frame}

\begin{frame}{Ex: Outarde de Macqueen - le présent}
  \begin{columns}
    \begin{column}[c]{0.5\textwidth}
      \includegraphics[width=.9\textwidth]{outarde_breeding}
    \end{column}
    \begin{column}[l]{0.5\textwidth}
      \begin{center}
        \includegraphics[width=\textwidth]{outarde_wintering} 
      \end{center}
    \end{column}
  \end{columns}
\end{frame}
\begin{frame}{Ex: Outarde de Macqueen - le futur}
  \begin{columns}
    \begin{column}[c]{0.5\textwidth}
      \includegraphics[width=\textwidth]{outarde_dpt_breeding}
    \end{column}
    \begin{column}[l]{0.5\textwidth}
      \begin{center}
        \includegraphics[width=\textwidth]{outarde_dpt_wintering} 
      \end{center}
    \end{column}
  \end{columns}
\end{frame}

\begin{frame}{Ex: Outarde de Macqueen - la migration}
  \begin{center}
    \includegraphics[width=.9\textwidth]{outarde_PlotMigrIBM_Uzbekistan_A6D6}
  \end{center}
\end{frame}



\subsection{Le Héron cendré}
% --------------------------------------

\begin{frame}
  \frametitle{Héron cendré : Biologie}
  \begin{columns}
    \begin{column}[c]{0.5\textwidth}
      \begin{itemize}
      \item \textit{Ardea cinerea} 
      \item Durée du vie : 25 ans 
      \item 1ère reproduction en général à 2 ans
      \end{itemize}

    \end{column}
    \begin{column}[l]{0.5\textwidth}
      \begin{center}
        \includegraphics[width=\textwidth]{heron} 
      \end{center}
    \end{column}
  \end{columns}
  
  
  
  \begin{itemize}

  \item Se reproduit au sein d’une colonie
  \item Essentiellement piscivore durant la reproduction
  \item Territorial pour l'aire d'alimentation
  \item La distance de l’aire de prospection alimentaire à la colonie ne peut dépasser 40 km (pour être compatible avec l’élevage des jeunes)
  \end{itemize}
\end{frame}

% 
\begin{frame}{Héron cendré : Biologie}
  \begin{center}
    Mécanisme de régulation de la colonies
  \end{center}
  \begin{columns}
    \begin{column}[c]{0.4\textwidth}
      \includegraphics[width=\textwidth]{colonieHeron}    
    \end{column}
    \begin{column}[c]{0.6\textwidth}
      \includegraphics[width=\textwidth]{saturationColonie}    
    \end{column}
  \end{columns}
  % 
\end{frame}




\begin{frame}
  \frametitle{Héron cendré : Distribution actuel}
  \begin{center}
    \includegraphics[width=0.8\textwidth]{distributionHeron} 
  \end{center}
  \begin{itemize}
  \item  La majeur partie de l’ancien continent sans l’Australasie, les déserts, et la plupart de l’Océanie
  \item Population mondiale : 236 000 – 281 000 couples nicheurs
  \item Population européenne : 150 000 – 180 000 couples nicheurs
  \end{itemize}
\end{frame}


\begin{frame}
  \frametitle{Héron cendré : Histoire}
  \begin{columns}
    \begin{column}[c]{0.6\textwidth}
      \begin{itemize}
      \item Considéré comme nuisible jusqu'en 1974
      \item 2 colonies au début du $XX^{ème}$ siècle
      \item Recolonisation depuis 1974 (espèce protégé)
      \item Progression lente des effectifs à la faveur des deux guerres
      \end{itemize}
    \end{column}
    \begin{column}[l]{0.4\textwidth}
      \includegraphics<1>[width=\textwidth]{heron2}    
      \includegraphics<2->[width=\textwidth]{colonieRefuge}    
    \end{column}
  \end{columns}
\end{frame}


\begin{frame}
  \frametitle{Héron cendré : Re-colonisation}
  \begin{columns}
    \begin{column}[c]{0.3\textwidth}
      \begin{itemize}[<+->]
      \item 1962
      \item 1974
      \item 1981
      \item 1989
      \item 1994
      \item 2000
      \end{itemize}
    \end{column}
    \begin{column}[l]{0.7\textwidth}
      \includegraphics<1>[width=\textwidth]{distribution1962}  
      \includegraphics<2>[width=\textwidth]{distribution1974} 
      \includegraphics<3>[width=\textwidth]{distribution1981} 
      \includegraphics<4>[width=\textwidth]{distribution1989} 
      \includegraphics<5>[width=\textwidth]{distribution1994} 
      \includegraphics<6>[width=\textwidth]{distribution2000} 
    \end{column}
  \end{columns}
\end{frame}




\begin{frame}{Objectif}
  \begin{itemize}[<+->]
  \item Modéliser le rôle du paysage dans le choix de la localisation des colonies d’une échelle locale à une échelle nationale
    \begin{itemize}
    \item Déterminer les zones favorables à l’espèces
    \item Modéliser la dynamiques spatio-temporelles
      \begin{itemize}
      \item Paramètres du paysage
      \item Inter-action entre les individus
      \end{itemize}
    \item Gérer le risque de conflits avec les pisciculteurs 
    \end{itemize}
  \end{itemize}
\end{frame}






\begin{frame}{La niche écologique}
  \begin{block}{Ecological Niche Factor Analysis (ENFA)}
    \begin{itemize}
    \item Méthode de modélisation des habitats potentiels
    \item Analyse factorielle ( proche de l’ACP ) de la niche écologiques
    \item Utilisation des seules données de présence de l’espèce pour la modélisation
    \item Caractérisation de la niche écologique de l’espèce étudiée par les variables écogéographiques prises en comptes
    \item Utilisation de la caractérisation de la niche écologique pour créer les cartes d’habitats favorables 
    \end{itemize}
  \end{block}
\end{frame}



\begin{frame}{Les variables éco-géographiques}
  Prise en compte de 20 Variables Ecogéographiques
  \vspace{20pt}
  \begin{columns}
    \begin{column}[c]{0.4\textwidth}
      \begin{itemize}[<+->]
      \item  Réseau hydrographique
      \item Paysage
      \item Topographie
      \item Perturbations anthropiques
      \end{itemize}
    \end{column}
    \begin{column}[c]{0.6\textwidth}
      \includegraphics[width=\textwidth]{photoVariableEco} 
    \end{column}
  \end{columns}
\end{frame}


\begin{frame}{Caractérisation de la niche}
  Caractérisation de la niche écologique par :
  \vspace{20pt}
  \begin{columns}
    \begin{column}[c]{0.5\textwidth}
      \begin{itemize}[<+->]
      \item  Importance des zones de marais
      \item La densité de canaux et de fossés
      \item Hétérogénéité des habitats
      \end{itemize}
    \end{column}
    \begin{column}[c]{0.5\textwidth}
      \includegraphics[width=.8\textwidth]{heron3} 
    \end{column}
  \end{columns}
\end{frame}



\begin{frame}{Sur la France}
  \begin{center}
    Les habitats potentiels sur l’ensemble du territoire:
    \vspace{10pt}
    \includegraphics[width=.5\textwidth]{HSFrance}
  \end{center}
\end{frame}



\begin{frame}{La grille}
  \begin{itemize}
  \item France découpée selon une grille  
  \item Chaque cellule 10 Km sur 10 Km
  \item Une cellule possèdent 2 attributs : 
    \begin{itemize}
    \item le nombre de colonies qu’elle contient (issu des recensement)
    \item une valeur de qualité d’habitat (HS)
    \end{itemize}
  \end{itemize}
\end{frame}

\begin{frame}{La grille : les rasters}
  \begin{center}
    \textcolor{red}{Préparation des données sous SIG}\\
  \end{center}
  \begin{columns}
    \begin{column}[c]{0.5\textwidth}
      \begin{center}
        \includegraphics[width=.95\textwidth]{grilleHS}
      \end{center}
    \end{column}
    \begin{column}[c]{0.5\textwidth}
      \begin{center}

        \includegraphics[width=.95\textwidth]{init}
      \end{center}
    \end{column}
  \end{columns}

\end{frame}

\begin{frame}{Paramètres de la dynamique}
  \includegraphics[width=.95\textwidth]{grilleParametreAutomate}
\end{frame}



\begin{frame}{Les règles d’installation}
  Les règles d'installations sont définit par analyse statistique (GLM): 
  \begin{columns}
    \begin{column}[c]{0.5\textwidth}
      \begin{block}{Loire-Atlantique}
        \begin{itemize}
        \item Pas d'effets des colonies et des effectifs
        \item Seul HS (Habitat Suitability) est conservé 
        \end{itemize}
      \end{block}

    \end{column}
    \begin{column}[c]{0.5\textwidth}
      \begin{block}{France}
        \begin{itemize}
        \item Présence de colonies à plus ou moindre proximité 
        \item Pas d'effet des effectifs
        \item Densité dépendance : les variables au carré possèdent des estimateurs négatifs
        \end{itemize}

      \end{block}
    \end{column}
  \end{columns}

  
\end{frame}



\begin{frame}{Dynamique spatio-temporelle}
  \begin{center}
    \includegraphics[width=\textwidth]{dynamiqueFitHeron}

    (a) 1985, (b) 1994, (c) 2001
  \end{center}
\end{frame}

\begin{frame}{Changement de comportement}
  \begin{center}
    \includegraphics[width=.6\textwidth]{nbColEtHS}

  \end{center}

\end{frame}




\begin{frame}{Conclusion}
  \begin{itemize}
  \item Projection de la dynamique futur  
  \item Définition des zones à risque d'interaction avec activitées humaines

  \end{itemize}

\end{frame}




\subsection{L'Ecureuil à ventre rouge}
\begin{frame}{L'Ecureuil à ventre rouge}
  \begin{center}
    \includegraphics[width=.7\textwidth]{squirrelPhoto1}
  \end{center}
\end{frame}

\begin{frame}{L'Ecureuil à ventre rouge}
  \begin{center}
    \includegraphics[width=.9\textwidth]{squirrel1}
  \end{center}
\end{frame}


\begin{frame}{L'espèce}
  \begin{center}
    \includegraphics[width=.9\textwidth]{squirrel2}
  \end{center}
\end{frame}



\begin{frame}{Dynamique de colonisation}
  \begin{center}
    \includegraphics[width=.9\textwidth]{squirrel3}
  \end{center}
\end{frame}


\begin{frame}{Structuration du modèle}
  \begin{center}
    \includegraphics[width=.9\textwidth]{squirrel4}
  \end{center}
\end{frame}


\begin{frame}{Discrétisation de l'habitat}
  \begin{center}
    \includegraphics[width=.9\textwidth]{squirrel5}
  \end{center}
\end{frame}


\begin{frame}{Estimations des capacités de charges}
  \begin{center}
    \includegraphics[width=.9\textwidth]{squirrel6}
  \end{center}
\end{frame}

\begin{frame}{Estimations des capacités de charges}
  \begin{center}
    \includegraphics[width=.9\textwidth]{squirrel7}
  \end{center}
\end{frame}


\begin{frame}{Estimations de la connectivité}
  \begin{center}
    \includegraphics[width=.9\textwidth]{squirrel8}
  \end{center}
\end{frame}


\begin{frame}{Estimations de la connectivité}
  \begin{center}
    \includegraphics[width=.9\textwidth]{squirrel9}
  \end{center}
\end{frame}


\begin{frame}{L'automate cellulaire}
  \begin{center}
    \includegraphics[width=.9\textwidth]{squirrel10}
  \end{center}
\end{frame}


\begin{frame}{L'automate cellulaire}
  \begin{center}
    \includegraphics[width=.9\textwidth]{squirrel11}
  \end{center}
\end{frame}


\begin{frame}{Dynamique de populations}
  \begin{center}
    \includegraphics[width=.9\textwidth]{squirrel12}
  \end{center}
\end{frame}


\begin{frame}{Projection}
  \begin{center}
    \includegraphics[width=.9\textwidth]{squirrel13}
  \end{center}
\end{frame}
\begin{frame}{Calibration}
  \begin{center}
    \includegraphics[width=.9\textwidth]{squirrel14}
  \end{center}
\end{frame}


\begin{frame}{Projection de l'expansion}
  \begin{center}
    \includegraphics[width=.9\textwidth]{squirrel15}
  \end{center}
\end{frame}


\begin{frame}{Scénario de gestion}
  \begin{center}
    \includegraphics[width=.9\textwidth]{squirrel16}
  \end{center}
\end{frame}


\begin{frame}{Effet de la capture sur la dynamique}
  \begin{center}
    \includegraphics[width=.9\textwidth]{squirrel17}
  \end{center}
\end{frame}


\begin{frame}{Effet des scénarios de capture}
  \begin{center}
    \includegraphics[width=.6\textwidth]{squirrel18}
  \end{center}
\end{frame}


\begin{frame}{L'autoroute une barrière valorisable}
  \begin{center}
    \includegraphics[width=.9\textwidth]{squirrel20}
  \end{center}
\end{frame}


\begin{frame}{Conclusion}
  \begin{center}
    \includegraphics[width=.9\textwidth]{squirrel19}
  \end{center}
\end{frame}

% =======================================================
\section{Dynamique des communautés}
% =======================================================


% ------------------------------------------------
\subsection{La connectivité des habitat d'une communauté herbacée}
% ------------------------------------------------
\begin{frame}{Les typologies des menaces}
  \begin{columns}
    \begin{column}[c]{0.6\textwidth}
      J. Diamond (1989) : \textit{Evil quartet}
      \vspace{10pt}
      \includegraphics<1>[width=\textwidth]{evilquartet1.png}
      \includegraphics<2>[width=\textwidth]{evilquartet2.png}
      \includegraphics<3>[width=\textwidth]{evilquartet3.png}
      \includegraphics<4>[width=\textwidth]{evilquartet4.png}
      \includegraphics<5->[width=\textwidth]{evilquartet5.png}
    \end{column}
    \begin{column}[l]{0.4\textwidth}
      \includegraphics[width=0.5\textwidth]{Diamond}    
    \end{column}
  \end{columns}
\end{frame}

\begin{frame}{Importance la connectivité}
  \begin{columns}
    \begin{column}[c]{0.6\textwidth}
      J. Diamond (1989) : \textit{Evil quartet}
      \includegraphics[width=\textwidth]{evilquartet2.png}
    \end{column}
    \begin{column}[l]{0.4\textwidth}
      \begin{itemize}
      \item Contexte de changements globaux\\
      \item Besoins des populations :
        \begin{itemize}
        \item dispersion
        \item installation
        \end{itemize}
      \item Conditionnés par : 
        \begin{itemize}
        \item qualité des habitats
        \item fragmentation
        \item connectivité 
        \end{itemize}
      \end{itemize}
    \end{column}
  \end{columns}
\end{frame}

\begin{frame}{La connectivité}
  \begin{block}{Définition}
    Capacité d’un paysage, à faciliter ou à limiter le déplacement des
    organismes entre deux patchs d'habitats 
    à travers la matrice paysagère (Taylor, 1993, Merriam 1984).
  \end{block}
  \pause
  \begin{alertblock}{Les mesures}
    Définition très vague $\rightarrow$ Pas de consensus sur les méthodes de mesure.
  \end{alertblock}
\end{frame}


\begin{frame}{Les mesures de connectivités du paysage}
  \textcolor{red}{\textbf{Pas de consensus}\\  }
  Classement des mesures en fonction des données utilisées \\ 
  \textit{{\scriptsize Calabrese et al.2004, : Minor et al. 2009}} \\
  \vspace{5pt}
  \begin{center}
    \includegraphics[width=0.75\textwidth]{mesureConnect}
  \end{center}
\end{frame}


\begin{frame}{Étude à l’échelle du paysage}
  \begin{columns}
    \begin{column}[c]{0.6\textwidth}
      \includegraphics<1>[width=\textwidth]{cartehabitatCBN1.png}
    \end{column}
    \begin{column}[c]{0.5\textwidth}
      \begin{itemize}
      \item Seine Saint Denis
      \item CBNBP \\ \textit{\tiny{Conservatoire Botanique National \\du Bassin Parisien}}
      \end{itemize}
      \begin{center}
        \includegraphics[width=0.3\textwidth]{CBNBP.png}
      \end{center}
    \end{column}
  \end{columns} 
\end{frame}

\begin{frame}{Données de connectivité \textit{a priori}}
  \begin{columns}
    \begin{column}[c]{0.6\textwidth}
      \includegraphics<1>[width=\textwidth]{cartehabitatCBN2.png}
      \includegraphics<2->[width=\textwidth]{cartehabitatCBN4.png}
    \end{column}
    \begin{column}[c]{0.5\textwidth}
      \begin{itemize}
      \item<1-> Cartographie des habitats: 446 milieux naturelles ouverts
        \begin{itemize}
        \item<1-> connectivité "distance"
        \end{itemize}
      \item<2-> Matrice de vert
        \begin{itemize}
        \item<2-> connectivité "chemin de moindre coût"
        \end{itemize}
      \item <3>HYPOTHÈSE: indice quantitatif standardisé corrélé à la
        perméabilité
      \end{itemize}
    \end{column}
  \end{columns} 
\end{frame}

\begin{frame}{Données de connectivité \textit{a posteriori}}
  \begin{columns}
    \begin{column}[c]{0.6\textwidth}
      \includegraphics[width=\textwidth]{cartehabitatCBN3.png}
    \end{column}
    \begin{column}[c]{0.5\textwidth}
      \begin{itemize}
      \item Inventaire floristique standardisé dans 21/446 patches
        \begin{itemize}
        \item distance taxonomique (Indice de Bray-Curtis)
        \end{itemize}
      \end{itemize}
    \end{column}
  \end{columns} 
\end{frame}

\begin{frame}{Les mesures de connectivités utilisées}
  \begin{columns}
    \begin{column}[c]{0.25\textwidth}
      \includegraphics<1>[width=0.95\textwidth]{connectivity1.png}
      \includegraphics<2>[width=0.95\textwidth]{connectivity2.png}
      \includegraphics<3>[width=0.95\textwidth]{connectivity3.png}
      \includegraphics<4>[width=0.95\textwidth]{connectivity4.png}
    \end{column}
    \begin{column}[c]{0.80\textwidth}
      \begin{itemize}
      \item 2 connectivités \textit{a priori}    
        \begin{itemize}
        \item<2-> \textbf{Connectivité "Distance"} \\
          définie par l'organisation spatiale des patchs.
        \item<3-> \textbf{Connectivité "Chemin de moindre coût"}\\
          prise en compte de la perméabilité du paysage.
        \end{itemize}
      \item  1 connectivité \textit{a posteriori}
        \begin{itemize}
        \item<4> \textbf{Connectivité réalisée} \\
          résultat de la connectivité: \\
          distance génétique, distance taxonomique.
        \end{itemize}
      \end{itemize}
    \end{column}
  \end{columns}
\end{frame}

\begin{frame}{Principe du modèle}
  \begin{center}
    Caractériser l'effet de la perméabilité sur la structuration des méta-communautés\\
    \vspace{10pt}
    \includegraphics[width=0.5\textwidth]{principeModelConnect.png}
    
  \end{center}
\end{frame}

\begin{frame}{Un modèle individus centrés}
  \begin{columns}[t]
    \begin{column}[c]{0.6\textwidth}
      \begin{itemize}[<+->]
      \item <1->Tenir compte de la perméabilité 
        \begin{itemize}
        \item <1-> information géographique très fine $15m \times 15m$
        \item <1-> le nombre de chemins possibles infini
        \end{itemize}
      \item <2->Modèle de dispersion de graines 
        \begin{itemize}
        \item <2->individus centrés 
        \item <2->probabiliste
        \item <2->chemin de moindre coût
        \end{itemize}
      \end{itemize}
    \end{column}
    \begin{column}[c]{0.4\textwidth}
      \includegraphics<1>[width=\textwidth]{graine1.png}
      \includegraphics<2>[width=\textwidth]{graine2.png}
      \includegraphics<3>[width=\textwidth]{graine3.png}      
      \includegraphics<4>[width=\textwidth]{graine4.png}
      \includegraphics<5>[width=0.8\textwidth]{IVconnexions.png}
    \end{column}
  \end{columns}
\end{frame}

\begin{frame}{Le modèle}
  \begin{center}
    \includegraphics<1>[width=.6\textwidth]{methodeConnect1.png}
    \includegraphics<2>[width=.6\textwidth]{methodeConnect2.png}
    \includegraphics<3>[width=.6\textwidth]{methodeConnect3.png}
    \includegraphics<4>[width=.6\textwidth]{methodeConnect4.png}
    \includegraphics<5>[width=.6\textwidth]{methodeConnect5.png}
    \includegraphics<6>[width=.6\textwidth]{methodeConnect6.png}
    \includegraphics<7>[width=.6\textwidth]{methodeConnect7.png}
  \end{center}
\end{frame}

\begin{frame}{Mesure de la force des connections}
  \textbf{Chemin le plus court} entre deux patchs\\
  \textit{Les autres patchs sont utilisés comme relais}\\
  \vspace{15pt}
  \begin{center}
    \includegraphics[width=0.8\textwidth]{indicReseau1.png}
  \end{center}
\end{frame}

\begin{frame}{Mesure de la force des connections}
  \textbf{Flux maximal} ente deux patchs\\
  \textit{Nombre de chemins indépendants (directs ou indirects)}\\
  \vspace{15pt}
  \begin{center}
    \includegraphics[width=0.8\textwidth]{indicReseau2.png}
  \end{center}
\end{frame}

\begin{frame}{Calibration}
  \begin{center}
    \includegraphics<1>[width=.75\textwidth]{calibration1}
    \includegraphics<2>[width=.75\textwidth]{calibration2}
  \end{center}
\end{frame}


\begin{frame}{Résultats}
  Meilleur réseau pour chaque connectivité \textit{a priori}\\
  \vspace{20pt}
  \includegraphics<1>[width=\textwidth]{resultatConnect1.png}
  \includegraphics<2>[width=\textwidth]{resultatConnect2.png}
\end{frame}


\begin{frame}{Conclusion}
  \begin{columns}
    \begin{column}[c]{0.55\textwidth}
      \includegraphics<1>[width=\textwidth]{cartehabitatCBN2.png}
      \includegraphics<2>[width=\textwidth]{cartehabitatCBN7.png}
      \includegraphics<3>[width=\textwidth]{cartehabitatCBN6.png}
    \end{column}
    \begin{column}[c]{0.6\textwidth}
      \begin{itemize}
      \item<1-> Importance de la perméabilité dans la structuration des communautés
      \item<2-> Chemin de moindre coût = bon prédicteur de la perméabilité
      \item<3-> Mise en évidence des barrières:	
        \begin{itemize}
        \item outil de  gestion
        \item mise en place de corridors écologiques
        \end{itemize}
      \end{itemize}
    \end{column}
  \end{columns}
\end{frame}


\begin{frame}{Et avec un automate cellulaire}
  \begin{center}
    \includegraphics<1>[width=.8\textwidth]{carteRomain2000m8couleurs}
    
  \end{center}

  
  
\end{frame}




\subsection{Oiseaux agricole, changements globaux et taux de migration}

\begin{frame}[Les oiseaux agricoles]
  \includegraphics[width=\textwidth]{transition}
\end{frame}


\begin{frame}{Indicateur européen et français}
  \begin{center}
    \includegraphics<1>[width=.75\textwidth]{indicEurope}
    \includegraphics<2>[width=.7\textwidth]{tendenceIndicateurOiseau}\\
    \includegraphics<2>[width=\textwidth]{spAgri}
  \end{center}
\end{frame}


\begin{frame}{Protocole STOC (Breeding Bird Survey)}
  \begin{center}
    STOC: \textcolor{red}{S}uivi \textcolor{red}{T}emporel des \textcolor{red}{O}iseaux \textcolor{red}{C}ommuns
  \end{center}
  \begin{columns}[c]
    \begin{column}[c]{0.5\textwidth}
      \begin{center}
        2300 carrés suivis au moins une fois entre 2001 et 2013 \\
        \includegraphics[width=.8\textwidth]{stoc_carre_2001-2013}
      \end{center}
    \end{column}
    \begin{column}[c]{0.5\textwidth}
      \begin{small}
        \begin{itemize}
        \item  Sites choisie aléatoirement
        \item 10 pts d'écoute de 5mn par site
        \item 2 passages pas an
        \item Description standardisée de l'hab.
        \end{itemize}
      \end{small}
      \begin{center}
        \includegraphics[width=.7\textwidth]{STOC_points}
      \end{center}
    \end{column}
  \end{columns}
  \begin{center}
    175 espèces suivies
  \end{center}
\end{frame}

\begin{frame}{Protocole STOC: Alouette des champs (\textit{Alauda arvensis})}
  \begin{columns}[c]
    \begin{column}[c]{0.5\textwidth}
      \includegraphics[width=.5\textwidth]{Alouette}
      \begin{itemize}[<+->]
      \item  Tendance :
        \begin{itemize}
        \item $-22\%$ depuis 1989, déclin
        \item $-10\%$ depuis 2001, diminution
        \end{itemize}
      \item Distribution géographique par espèce
      \end{itemize}
    \end{column}
    \begin{column}[c]{0.5\textwidth}
      \begin{center}
        \includegraphics<1-3>[width=.9\textwidth]{tendanceAlouette}
        \includegraphics<4->[width=.9\textwidth]{distributionAlouette} 
      \end{center}
    \end{column}
  \end{columns}
\end{frame}


\begin{frame}{Suivre ou ne pas suivre sa niche climatique…}
  \begin{center}
    \includegraphics[width=.9\textwidth]{CTI}
  \end{center}

  \begin{tiny}
    Devictor et al. 2008
  \end{tiny}
\end{frame}

\begin{frame}{Suivre ou ne pas suivre sa niche climatique…}
  \begin{center}
    \includegraphics[width=.6\textwidth]{CTI_lat}
  \end{center}

  \begin{tiny}
    Devictor et al. 2008
  \end{tiny}
\end{frame}

\begin{frame}{Suivre ou ne pas suivre sa niche climatique…}
  \begin{center}
    \includegraphics[width=.6\textwidth]{CTIevolution}
  \end{center}

  \begin{tiny}
    Devictor et al. 2008
  \end{tiny}
\end{frame}





\begin{frame}{L'agriculture}
  \begin{columns}[c]
    \begin{column}[c]{0.4\textwidth}
      \begin{center}
        En Europe: \\ Agriculture $\approx 50\%$  surface 
      \end{center}
    \end{column}
    \begin{column}[c]{0.6\textwidth}
      \begin{center}
        \includegraphics[width=0.9\textwidth]{agriEurope}
      \end{center}
      \begin{tiny}
        Temme et Verburg, 2011
      \end{tiny}
    \end{column}
  \end{columns}
\end{frame}

\begin{frame}{Evolution paysage agricole}
  \begin{center}
    Evolution des paysages agricoles au cours des dernières décennies \\
    \vspace{10pt}
    \includegraphics[width=\textwidth]{evolutionPaysageAgri}
    \vspace{10pt}
    \textcolor{red}{Homogénéisation du paysage et des pratiques}
  \end{center}
\end{frame}





\begin{frame}{Objectif}
  \begin{itemize}[<+->]
  \item Comment les espèces déplacent leur aire de distribution
    \begin{itemize}
    \item scénario agricole
    \item Outil d'aide à la décision
    \end{itemize}
  \end{itemize}
\end{frame}

\begin{frame}{Hypothèses}
  \begin{itemize}[<+->]
  \item Capacité de déplacement des aires très difficile à estimé
    \begin{itemize}
    \item proxy : taux d'échange d'individus = taux de migration    
    \end{itemize}
  \item proposition d'un modèle mécaniste pour l'estimer
  \end{itemize}
\end{frame}





\begin{frame}{Petites Région Agricoles}
  
  \begin{columns}[c]
    \begin{column}{.65\textwidth}
      \begin{center}
        
        \includegraphics[width=\textwidth]{pra}
      \end{center}
    \end{column}
    \begin{column}{.45\textwidth}



      713 Small Agricultural Region ($PRA$)

      \begin{itemize}
      \item formes variables
      \item tailles variables
      \end{itemize}
      
    \end{column}
  \end{columns}
\end{frame}




\begin{frame}{Meta-communauté face aux perturbations}
  \begin{columns}
    \begin{column}[c]{0.45\textwidth}
      \begin{center}
        \includegraphics<1>[width=\textwidth]{metapopulations0}
        \includegraphics<2>[width=\textwidth]{metapopulations1}
        \includegraphics<3>[width=\textwidth]{metapopulations2}
        \includegraphics<4>[width=\textwidth]{metapopulations3}
        \includegraphics<5->[width=\textwidth]{metapopulations4}
      \end{center}
    \end{column}
    \begin{column}[c]{0.6\textwidth}
      \begin{small}
        \begin{itemize}[<+->]
        \item Dynamique locales en fonction de variable locales
        \item Taux de dispersion essentiel
        \end{itemize}
      \end{small}
    \end{column}
  \end{columns}
\end{frame}



\begin{frame}{Modèle mécaniste de metapop}
  \begin{footnotesize}
    \begin{equation}
      N_{i,j}(t+1) = N_{i,j}(t) + r_{i}N_{i,j}(t)\left(1-\frac{N_{i,j}(t)}{\textcolor{blue}{K_{i,j}(t)}} \right)+\textcolor{red}{\bm{\tau_i}}\textcolor{orange}{\left( N_{ij}(t) - \sum_{\scriptsize j' \mbox{ close to }j} \hspace{-0.3 cm} N_{ij'}(t) \right)} 
      \label{eq_logistic_abondance_linear}
    \end{equation}
    \begin{itemize}
    \item $r_{i}$ taux de croissance intrinsèque
    \item $\textcolor{blue}{K_{i,j}(t)}$ capacité de charge
    \item $\textcolor{red}{\tau_i}$ taux de dispersion
    \end{itemize}
  \end{footnotesize}
\end{frame}

\begin{frame}{Abundance $N$}
  \begin{columns}[c]
    \begin{column}{.7\textwidth}
      \includegraphics<1>[width=\textwidth]{praSTOCspecies1}  
      \includegraphics<2>[width=\textwidth]{praSTOCspecies2} 
      \includegraphics<3>[width=\textwidth]{praSTOCspecies3}      
    \end{column}
    \begin{column}{.3\textwidth}
      37 species 
      \begin{itemize}[<+->]
      \item 23 farmland bird species 
      \item 14 generalist birds species

      \end{itemize}          
    \end{column}
  \end{columns}
  
\end{frame} 




\begin{frame}{$\tau$ : taux de migration}
  \begin{center}
    \includegraphics<1>[width=0.9\textwidth]{exemplePRA}
    \includegraphics<2>[width=0.9\textwidth]{exemplePRABuff10000}
    \includegraphics<3>[width=0.9\textwidth]{exemplePRABuff50000}
  \end{center}
\end{frame} 


\begin{frame}{$K$ : capacité de charge}
  13 variables temp, precip, ...
  \begin{center}
    \includegraphics<1>[width=0.8\textwidth]{dataClimat}
    \includegraphics<2>[width=0.8\textwidth]{praClimat}
    \includegraphics<3>[width=0.8\textwidth]{pcaBiplotClimat}
    \includegraphics<4>[width=0.8\textwidth]{pcaAxe1Climat}
  \end{center}
\end{frame} 

\begin{frame}{$R$ : taux de croissance}
  \begin{small}
    \begin{equation}
      R_i(t) = \sum_{u=1}^7 a_u A_{iu}(t)
      \label{eq_R}
    \end{equation}
  \end{small}
\end{frame}

\begin{frame}{$K$ : capacité de charge}
  OTEX: orientation téchnico-économique $\Rightarrow$ 7 variables de types d'occupation du sol
  
  \begin{itemize}
  \item CROP 
  \item INDUS\_CROP 
  \item OTHER\_CROP 
  \item PERENIAL
  \item LIVESTOCK
  \item POLYCULTURE
  \item GRANIVORE
  \end{itemize}


\end{frame} 



\begin{frame}{Indice de Spécialisation des espèces (SSI)}
  \begin{columns}[c]
    \begin{column}[c]{0.55\textwidth}
      \begin{center}
        \includegraphics<1-2>[width=\textwidth]{SSI1}
        \includegraphics<3->[width=\textwidth]{SSI2}
      \end{center}
    \end{column}
    \begin{column}[c]{0.45\textwidth}
      \begin{center}
        Coefficient de variation\\de l'abondance parmi \\les habitats
        \begin{equation}
          SSI_i = \frac{\sqrt{\frac{\sum_{j=1}^K(d_j-\bar{d})^2}{K-1}}}{\bar{d}}
        \end{equation}
      \end{center}
      \begin{itemize}
      \item<2-> Une espèce agricole:\\$SSI_{\textit{Skylark}} = 1.07$
      \item<3>  Une espèce généraliste: \\$SSI_{\textit{Blackbird}} = 0.25$
      \end{itemize}

    \end{column}
  \end{columns}
\end{frame}






\begin{frame}{National trends , $\tau$ et SSI }
  
  \begin{center}
    \includegraphics[width=.7\textwidth]{trend_buffeur_allSP_specialisation}\\
    \textcolor{gray}{Gray} : Generalist species |  \textcolor{orange}{Orange} : Farmland specialist species
    
  \end{center}
  
  
  
\end{frame} 



\begin{frame}{$\tau$ et SSI}
  
  
  \begin{center}
    \includegraphics[width=.7\textwidth]{b_SSI_allSP_2}\\
    \textcolor{gray}{Gray} : Generalist species |  \textcolor{orange}{Orange} : Farmland specialist species
    
    
  \end{center}


\end{frame} 






\begin{frame}{Conclusion}
  \begin{itemize}[<+->]
  \item importance de la migration inter patch dans la dynamique local (30 sp / 37)
  \item les populations souffre globalement moins si elle ont un $\tau$ fort
  \item sauf si elle sont très spécialistes
  \item $\rightarrow$ effet aggravant la réponse des spécialistes
  \item $\tau$ semble avoir un sens fonctionnel au vu de sa corrélation avec le HRS
  \end{itemize}
\end{frame}





% =======================================================
\section{Conclusions, perspectives}
% =======================================================

\begin{frame}{Conclusion et perspectives}
  \begin{itemize}
  \item Nombreux compromis quant à la portée et la pertinence des résultats\\
  \end{itemize}
  \begin{center}
    \includegraphics[width=.9\textwidth]{conclu1.png}
  \end{center}
\end{frame}

\begin{frame}{Conclusion et perspectives}
  \begin{itemize}
  \item Nombreux compromis quant à la portée et la pertinence des résultats\\
  \item Données
  \end{itemize}
  \begin{center}
    \includegraphics[width=.9\textwidth]{conclu2.png}
  \end{center}

\end{frame}

\begin{frame}{Conclusion et perspectives}
  \begin{itemize}
  \item Nombreux compromis quant à la portée et la pertinence des résultats\\
  \item Données
  \item Intégration de la variabilité inter-individuelle et environnementale
  \item intégration des interactions inter-spécifiques
  \end{itemize}
\end{frame}


\begin{frame}[plain]
  \begin{center}
    \begin{huge}
      Merci ....
      
    \end{huge}

  \end{center}

  \includegraphics[width=\textwidth]{angry_birds_merci}
\end{frame}




\end{document}
