\documentclass[10pt]{beamer}
\usepackage[utf8]{inputenc}
%\usepackage[T1]{fontenc}
\usepackage[french]{babel}
%\usepackage{enumitem}
% \usepackage{natbib}
%\usepackage{hyperref}
% \usepackage{animate}
%\usepackage{graphicx}
%\usepackage{color}

\pdfminorversion=5 
\pdfcompresslevel=9
\pdfobjcompresslevel=3

%% Theme
% \useinnertheme[shadow=true]{rounded}
% \useoutertheme{shadow}
% \usecolortheme{orchid}
% \usecolortheme{whale}%\usetheme{progressbar}
%\usetheme[]{Berlin}

\useoutertheme[footline=authorinstitutetitle]{miniframes}
\useoutertheme{smoothtree}
\useinnertheme[shadow=true]{rounded}
\usecolortheme{orchid}
\usecolortheme{whale}


\graphicspath{{./images/}} 

% \usecolortheme{progressbar}
% \usefonttheme{progressbar}
% \useinnertheme{progressbar}
% \useoutertheme{progressbar}
%% Logo
% \logo{\includegraphics[height=0.5cm]{./images/Logo.png}}
%% Fond de la page
% \setbeamercolor{background canvas}{bg=couleur}

\title[Introduction à l'écologie]{Introduction à l'écologie}

\author{Romain Lorrillière | \textit{romain.lorrilliere@mnhn.fr}}
\institute{ONIRIS : DIE Santé de la faune sauvage non captive (Nov 2021)}
\date{23 Novembre 2021}

\AtBeginSection[]{
  \begin{frame}
   \frametitle{\insertsectionhead}
    \scriptsize \tableofcontents[currentsection,hideothersubsections]
  \end{frame}
}


%\AtBeginSubsection[]{
%  \begin{frame}
%    \frametitle{\insertsubsectionhead}
%    \scriptsize \tableofcontents[sectionstyle=show/shaded,subsectionstyle=show/shaded/hide ]
%  \end{frame}
%}
%\addtobeamertemplate{footline}{\insertframenumber/\inserttotalframenumber}

\begin{document}
\maketitle



% \section{Sommaire}
% =====================
 \begin{frame}
   \frametitle{Sommaire}
   \tableofcontents[hideallsubsections]
 \end{frame}




% ===============================================
\section{L'écologie}
% ===============================================

\begin{frame}{L'écologie, définition}
  \begin{center}
    \begin{block}{Ecologie}
      Science qui étudie les interactions des êtres vivants entre eux et avec leur milieu
    \end{block}
  
  \end{center}
\end{frame}



% ----------------------------------------
\subsection{Les grandes définitions}
% ----------------------------------------


\begin{frame}{Les facteurs écologiques}
  \begin{columns}
    \begin{column}[c]{0.5\textwidth}
     \begin{block}<1->{Facteurs biotiques}
       \begin{itemize}
        \item ressources alimentaires
        \item les interactions
          \begin{itemize}
          \item prédation
          \item parasitisme
          \item compétition
          \item symbiose
          \item ...
          \end{itemize}
        \end{itemize}
    \end{block}
   \end{column}
    \begin{column}[c]{0.5\textwidth}
      \begin{block}<2->{Facteurs abiotiques}
        L'ensemble des facteurs physico-chimiques
        \begin{itemize}
        \item air
        \item salinité
        \item sol
        \item température
        \item humidité
        \item lumière
        \item eau
        \item minéraux
        \item pH
        \end{itemize}
    \end{block}
    \end{column}
  \end{columns}
\end{frame}




\begin{frame}{La biodiversité}
  \begin{center}
    \begin{exampleblock}{Biodiversité}
   La diversité des écosystèmes, des espèces et des gènes dans l'espace et dans le temps, ainsi que les interactions au sein de ces niveaux d'organisation et entre eux.
 \end{exampleblock}
 \end{center}
\end{frame}


\begin{frame}{Des objets imbriqués} 
  \begin{columns}
    \begin{column}[c]{0.6\textwidth}
      \begin{itemize}[<+->]
      \item biome
      \item biosphère
      \item communauté
      \item ecosystème
      \item population
      \end{itemize}
    \end{column}
    \begin{column}[c]{0.4\textwidth}
      \begin{center}
       \includegraphics<6>[width=\textwidth]{biosphere_population}
      \end{center}
    \end{column}
  \end{columns}
 % \tiny{\cite{ref_biblio}}
\end{frame}



\begin{frame}{Des objets imbriqués} 
  \begin{columns}
    \begin{column}[c]{0.25\textwidth}
      \begin{small}
        \begin{block}<1->{A}
          \begin{enumerate}
          \item biome
          \item biosphère
          \item communauté
          \item ecosystème
          \item population
          \end{enumerate}
        \end{block}
      \end{small}
    \end{column}
    \begin{column}[c]{0.25\textwidth}
      \begin{small}
        \begin{block}<2->{B}
          \begin{enumerate}
          \item biosphère
          \item biome
          \item ecosystème
          \item communauté
          \item population
          \end{enumerate}
        \end{block}
      \end{small}
    \end{column}
    \begin{column}[c]{0.25\textwidth}
      \begin{small}
        \begin{block}<3->{C}
          \begin{enumerate}
          \item biosphère
          \item ecosystème
          \item biome
          \item population
          \item communauté
          \end{enumerate}
        \end{block}
      \end{small}
    \end{column}
    \begin{column}[c]{0.25\textwidth}
      \begin{center}
        \includegraphics[width=\textwidth]{biosphere_population}
      \end{center}
    \end{column}
  \end{columns}
  % \tiny{\cite{ref_biblio}}
\end{frame}

  

\begin{frame}{Des objets imbriqués} 
  \begin{columns}
    \begin{column}[c]{0.25\textwidth}
      \begin{small}
        \begin{block}{A}
          \begin{enumerate}
          \item biome
          \item biosphère
          \item communauté
          \item ecosystème
          \item population
          \end{enumerate}
        \end{block}
      \end{small}
    \end{column}
    \begin{column}[c]{0.25\textwidth}
      \begin{small}
        \begin{exampleblock}{B}
          \begin{enumerate}
          \item biosphère
          \item biome
          \item ecosystème
          \item communauté
          \item population
          \end{enumerate}
        \end{exampleblock}
      \end{small}
    \end{column}
    \begin{column}[c]{0.25\textwidth}
      \begin{small}
        \begin{block}{C}
          \begin{enumerate}
          \item biosphère
          \item ecosystème
          \item biome
          \item population
          \item communauté
          \end{enumerate}
        \end{block}
      \end{small}
    \end{column}
    \begin{column}[c]{0.25\textwidth}
      \begin{center}
        \includegraphics[width=\textwidth]{biosphere_population}
      \end{center}
    \end{column}
  \end{columns}
  % \tiny{\cite{ref_biblio}}
\end{frame}

  
\begin{frame}{Des objets imbriqués} 
  \begin{columns}
    \begin{column}[c]{0.7\textwidth}
      \begin{footnotesize}
     \begin{block}<1->{}
      1-\textbf{Biosphère} Ensemble des oganismes vivants et leurs milieux de vie.
    \end{block}
     \begin{block}<2->{}
       2-\textbf{Biome} Zone de vie majeure, c'est-à-dire les ensembles biologiques les plus larges que l'on puisse discerner à l'échelle des continents.     
     \end{block}
         \begin{block}<3->{}
     3-\textbf{Ecosystème} Ensemble formé par une \textit{communauté} d'êtres
     vivants en interrelation (\textit{biocénose}) avec son environnement
     (\textit{biotope}).
    \end{block}
      \begin{block}<4->{}
      4-\textbf{Commnauté} Une association écologique d'êtres vivants
    \end{block}

   \begin{block}<5->{}
      5-\textbf{Population} Groupe d'individus de la même \textbf{espèce} en interaction.
    \end{block}
\end{footnotesize}
  \end{column}
    \begin{column}[c]{0.3\textwidth}
      \begin{center}
         \includegraphics<1>[width=0.5\textwidth]{biosphere_population_1}
         \includegraphics<2>[width=0.5\textwidth]{biosphere_population_2}
         \includegraphics<3>[width=0.5\textwidth]{biosphere_population_3}
         \includegraphics<4>[width=0.5\textwidth]{biosphere_population_4}
         \includegraphics<5>[width=0.5\textwidth]{biosphere_population}
        \vspace{1cm}
        \includegraphics<1>[width=\textwidth]{global_biosphere_wikipedia}
         \includegraphics<2>[width=\textwidth]{biome-map-fr}
         \includegraphics<3>[width=\textwidth]{ecosystem-main-cropped}
         \includegraphics<4>[width=\textwidth]{camargue}
         \includegraphics<5>[width=\textwidth]{albatros}
      \end{center}
    \end{column}
  \end{columns}
 % \tiny{\cite{ref_biblio}}
\end{frame}



\begin{frame}{Les espèces} 
  \begin{columns}
    \begin{column}[c]{0.55\textwidth}
      \begin{block}{Espèce}
	2 grands concepts: 
        \begin{itemize}[<+->]
        \item Groupe d'organismes vivants qui partagent des caractères communs
          et qui peuvent se reproduire, échanger des gènes et
          produire une déscendence viable et féconde.
        \item Unité de base de la classification du vivant
     \begin{itemize}[<+->]
        \item L'espece est ici un taxon, un groupe monophylétique (ayant un ancêtre commun)
	\item possède un histoire évolutive commune (ligniée évolutive)
           \end{itemize}
           \end{itemize}
      \end{block}
    \end{column}
    \begin{column}[c]{0.45\textwidth}
      \begin{center}
        \includegraphics<1>[width=.9\textwidth]{faune-et-flore}     
        \includegraphics<2>[width=.9\textwidth]{Taxonomic_hierarchy_wikipedia}
        \includegraphics<3->[width=.9\textwidth]{taxon}
      \end{center}
    \end{column}
  \end{columns}
  \tiny{\cite{Mora2011}}
\end{frame}



\begin{frame}{C'est pas si simple} 
  \begin{columns}
    \begin{column}[c]{0.5\textwidth}
      \begin{itemize}[<+->]
        \item Les espèces domestiquées
      \item Les hybrides
        \begin{itemize}
        \item Ligron
        \item Tigron
 	\item $\rightarrow$ Règle d'Haldane : sexe hétérozigote stérile
        \end{itemize}
      \item Les espèces clinales (ou annulaires)\\ex. Salamandre \textit{Ensatina}
        \begin{itemize}
        \item Repro possible entre \textit{E. eschschltzii} et
          \textit{E. xanthoptica} ?
          \item $\rightarrow$ \textbf{OUI}
       \item Repro possible entre \textit{E. eschschltzii} et
          \textit{E. croceater} ?
          \item $\rightarrow$ \textbf{NON}
        \end{itemize}

      \end{itemize}
    \end{column}
    \begin{column}[c]{0.5\textwidth}
      \begin{center}
        \includegraphics<1>[width=.8\textwidth]{dogs}      
        \includegraphics<2-5>[width=.8\textwidth]{hybrides}
        \includegraphics<6->[width=\textwidth]{Wake_2016_fig6_RingSpecies_Ensatina}
      \end{center}
    \end{column}
  \end{columns}
  % \tiny{\cite{ref_biblio}}
\end{frame}


\begin{frame}{Les espèces} 
\begin{exampleblock}{Espèces}
\begin{itemize}[<+->]
\item  la notion d'espèce est une construction,\\ les deux concepts ne sont pas toujours consiliables
	\item  souvent l'unité de base de conservation de la biodiversité
  \end{itemize}
\end{exampleblock}
  \end{frame}


\begin{frame}{Les espèces} 
  \begin{columns}
    \begin{column}[c]{0.55\textwidth}
      \begin{block}{Espèce}
        \begin{itemize}[<+->]
         \item env 2 millions décrites et au moins 8 millions d'espèces
          eukaryotes
        \item bcp d'espèces d'insectes
	\item mais aussi grande diversité génétique chez les plantes (polyploïdie)
        \end{itemize}
      \end{block}
    \end{column}
    \begin{column}[c]{0.45\textwidth}
      \begin{center}
        \includegraphics<1>[width=.9\textwidth]{Mora_et_al_2011_fig2_nb_sp}
        \includegraphics<2>[width=.9\textwidth]{piechart_sp}
        \includegraphics<3->[width=.9\textwidth]{genetic_diversity_banana}
      \end{center}
    \end{column}
  \end{columns}
  \tiny{\cite{Mora2011}}
\end{frame}



% ========================================
\section{Un processus évolutif}
% ========================================


\begin{frame}{Un processus} 
  \begin{columns}
    \begin{column}[c]{0.6\textwidth}
     \begin{itemize}[<+->]
         \item Comprendre la biodiversité non comme un objet mais comme un
      processus
        \item l'évolution n'est ni bonne ni mauvaise
	\item l'évolution n'a pas de but (attention à l'"Intelligent design")
        \end{itemize}   
    \end{column}
    \begin{column}[c]{0.4\textwidth}
      \begin{center}
       \includegraphics[width=.5\textwidth]{flamme}     
      \end{center}
    \end{column}
  \end{columns}
 % \tiny{\cite{ref_biblio}}
\end{frame}





\begin{frame}{Un processus} 
  \begin{columns}
    \begin{column}[c]{0.35\textwidth}
      Un processus qui date de plus de 2 milliards d'années
    \end{column}
    \begin{column}[c]{0.65\textwidth}
      \begin{center}
        \includegraphics[width=\textwidth]{Spiral-diagram_biodiversity_history}
       \end{center}
    \end{column}
  \end{columns}
 % \tiny{\cite{ref_biblio}}
\end{frame}





% ----------------------------------------
\subsection{La séléction naturelle}
% ----------------------------------------


\begin{frame}{La séléction naturelle} 
  \begin{columns}
    \begin{column}[c]{0.5\textwidth}
      \includegraphics[width=0.3\textwidth]{Darwin}\\~\\
      1859 \textbf{Charles Darwin} publie \\
      \textit{The Origin of species}
   \begin{itemize}[<+->]
     \item les caractères des individus varient au sein d'une population
     \item héritabilité des caractères
     \item séléction des individus qui ont les caractères les plus
       favorables
       \begin{itemize}
       \item selection pour les ressources 
       \item séléction pour la reproduction (séléction sexuelle)
       \end{itemize}
     \end{itemize}
    \end{column}
    \begin{column}[c]{0.5\textwidth}
      \begin{center}
        \includegraphics<1-5>[width=\textwidth]{darwin_tree_1}
        \includegraphics<6>[width=\textwidth]{Charles-Darwin-tree-of-Life}    
      \end{center}
    \end{column}
  \end{columns}
  
  \tiny{\cite{Atzmon2015}}
\end{frame}



\begin{frame}{Force évolutive} 
  \begin{columns}
    \begin{column}[c]{0.65\textwidth}
      Forces susceptibles de modifier les fréquences de gènes dans les
      populations
      \begin{itemize}[<+->]
      \item \textbf{Mutation}: Changement dans la structure d’un gène, qui peut se
        transmettre à la génération suivante (ou pas)
      \item \textbf{Migration}: transmission de la variation génétique d’une
        population à une autre
      \item \textbf{Sélection}: Contribution différentielle des génotypes en fonction de
        leurs qualités relatives
      \item \textbf{Dérive}: Fluctuation stochastique des fréquences
        alléliques
        \begin{itemize}
        \item \textit{Goulot d'étranglement}, ex Guépard il y a 12000 ans
        \end{itemize}
      \end{itemize}
    \end{column}
    \begin{column}[c]{0.35\textwidth}
      \begin{center}
        \includegraphics<1>[width=\textwidth]{mutation}
        \includegraphics<2>[width=.8\textwidth]{gene_flow}
        \includegraphics<3>[width=\textwidth]{selection}
        \includegraphics<4>[width=\textwidth]{genetic_drift}
        \includegraphics<5>[width=\textwidth]{bottleneck_Cheetah}  
      \end{center}
    \end{column}
  \end{columns}
 % \tiny{\cite{ref_biblio}}
\end{frame}



% ----------------------------------------
\subsection{L'arbre et le réseau du vivant}
% ----------------------------------------



\begin{frame}{L'arbre du vivant}
  \begin{center}
    \includegraphics[width=\textwidth]{the-great-tree-of-life}
  \end{center}
\end{frame}


\begin{frame}{Le réseau du vivant}
  \begin{center}
     \includegraphics[width=.6\textwidth]{Doolittle_Web_of_Life}
   \end{center}
   Transfert horizontal de gène, entre espèces, par hybridation ou par
   l'intermédiaire des bactérie et des virus\\
   Ex. Homme moderne et homme de Néandertal, Mitochondrie
 \end{frame}

 
 \begin{frame}{L'arbre du vivant}
   Attention à la simplification en nombre d'espèces.\\
   Les insectes s'isolent rapidement génétiquement contrairement aux plantes. 
  \begin{center}
        \includegraphics<1>[width=.5\textwidth]{piechart_sp}
       \includegraphics<2>[width=\textwidth]{the-great-tree-of-life}
  \end{center}
\end{frame}




% ========================================
\section{Habitat et niche écologique}
% ========================================


% ----------------------------------------
\subsection{Les biomes}
% ----------------------------------------

\begin{frame}{Les biomes}
  \begin{center}
     \includegraphics[width=.7\textwidth]{biome-map-fr}
   \end{center}
\end{frame}



% ----------------------------------------
\subsection{Les habitats}
% ----------------------------------------


\begin{frame}{Les habitats} 
  \begin{columns}
    \begin{column}[c]{0.6\textwidth}
      \begin{itemize}[<+->]
      \item Lieu (petit ou grand) de la biosphère possédant des
        caractérisques biotique et abiotique = écosystème
        \begin{itemize}
        \item forêt
        \item rivière
        \item tronc d'arbre
        \end{itemize}
      \item les espèces sont adaptées à un ou plusieurs habitats
        \begin{itemize}
        \item specialistes 
        \item généralistes
        \end{itemize}
      \item par la séléction naturelle les espèces sont adaptées
        pour survire et se reproduire malgré les contraintes
      \item Mais \textbf{homogénéisation biotique}
      \end{itemize}
    \end{column}
    \begin{column}[c]{0.4\textwidth}
      \begin{center}
        \includegraphics<1-2>[width=\textwidth]{foret}
        \includegraphics<3>[width=\textwidth]{river}
        \includegraphics<4-5>[width=\textwidth]{tree}
        \includegraphics<6>[width=\textwidth]{pufin}
        \includegraphics<7>[width=\textwidth]{rat}
        \includegraphics<8>[width=\textwidth]{manchot_empereur}
        \includegraphics<9>[width=\textwidth]{clavel-et-al-2010-worldwide-decline-of-specialist-species-toward-a-global-functional-homogenization-l}
        \includegraphics<9>[width=\textwidth]{clavel-et-al-2010-worldwide-decline-of-specialist-species-toward-a-global-functional-homogenization-l}
        \includegraphics<10>[width=\textwidth]{SSIfragmentation}
        \includegraphics<11>[width=\textwidth]{tendenceIndicateurOiseau}
      \end{center}
    \end{column}
  \end{columns}
  % \tiny{\cite{ref_biblio}}
\end{frame}



\begin{frame}{Répartition} 
  \begin{columns}
    \begin{column}[c]{0.4\textwidth}
      Région géographique dans laquelle vit une espèce
      \begin{itemize}[<+->]
      \item large répartition \\ Ex: Loup gris
      \item petite répartition \\ Ex: Sitelle corse
        \begin{itemize}
        \item endemisme
        \end{itemize}
      \item migration
          \begin{itemize}
          \item courte distance
          \item longue distance
        \end{itemize}
      \end{itemize}
      
    \end{column}
    \begin{column}[c]{0.6\textwidth}
      \begin{center}
        \includegraphics<1>[width=.75\textwidth]{carte-loup}
        \includegraphics<1>[width=.5\textwidth]{loups}
        \includegraphics<2-3>[width=.5\textwidth]{Sitta_whiteheadi_map}
        \includegraphics<2-3>[width=.3\textwidth]{sittelle_corse}
        \includegraphics<4>[width=\textwidth]{bird_migration}
        \includegraphics<5>[width=.6\textwidth]{migrationCourt}
        \includegraphics<6>[width=.6\textwidth]{migrationLong}
        
      \end{center}
    \end{column}
  \end{columns}
 % \tiny{\cite{ref_biblio}}
\end{frame}




\begin{frame}{Les facteurs écologiques}
  \begin{columns}
    \begin{column}[c]{0.5\textwidth}
     \begin{block}{Facteurs biotiques}
       \begin{itemize}
        \item ressources alimentaires
        \item les interactions
          \begin{itemize}
          \item prédation
          \item parasitisme
          \item compétition
          \item symbiose
          \item ...
          \end{itemize}
        \end{itemize}
    \end{block}
   \end{column}
    \begin{column}[c]{0.5\textwidth}
      \begin{block}{Facteurs abiotiques}
        L'ensemble des facteurs physico-chimiques
        \begin{itemize}
        \item air
        \item salinité
        \item sol
        \item température
        \item humidité
        \item lumière
        \item eau
        \item minéraux
        \item pH
        \end{itemize}
    \end{block}
    \end{column}
  \end{columns}
\end{frame}



% ----------------------------------------
\subsection{Les niches écologiques}
% ----------------------------------------

\begin{frame}{Les niches écologiques}
  \begin{columns}
    \begin{column}[c]{0.5\textwidth}
      \begin{itemize}[<+->]
      \item Les facteurs limitants d'une espèce, sur tous les
        gradients environnementaux
        \begin{itemize}
        \item Niche optimale
        \item Niche secondaire
          \begin{itemize}
          \item Exemple du Héron cendré
          \end{itemize}
        \end{itemize} 
      \item la place de l'espèce au sein de la communauté
        (compétition) et de
        l'écosystème (autres intercations)
      \item niche fondamentale Vs niche réalisée
      \item Exclusion compétitve
      \end{itemize}
    \end{column}
    \begin{column}[c]{0.5\textwidth}
      \begin{center}
        \includegraphics<1-2>[width=\textwidth]{niche1}
        \includegraphics<3>[width=\textwidth]{niche2}
        \includegraphics<4>[width=0.5\textwidth]{heron}
        \includegraphics<4>[width=.75\textwidth]{nbColEtHS}
        \includegraphics<5>[width=\textwidth]{niche3}
        \includegraphics<6>[width=.75\textwidth]{Schematisation-theorique-dune-niche-ecologique-avec-la-niche-ecologique-fondamentale}
        \includegraphics<7->[width=\textwidth]{marnage}
      \end{center}
    \end{column}
  \end{columns}
  \tiny{\cite{Lorrilliere2010}}
\end{frame}

  

% ----------------------------------------
\subsection{Les niches climatiques}
% ----------------------------------------

\begin{frame}
  \begin{columns}
    \begin{column}[c]{.5\textwidth}
      \begin{itemize}[<+->]
      \item Contexte de changement climatique
      \item La niche climatique détermine où une espèce vit
      \item et des projections détermineront comment elle réagira aux changements climatiques au fil du temps
      \end{itemize}
    \end{column}
    \begin{column}[c]{.5\textwidth}
      \begin{center}
      \includegraphics<1>[width=.7\textwidth]{changement_climatique}
      \includegraphics<1>[width=.7\textwidth]{changement_climatique_model}
      \includegraphics<2>[width=\textwidth]{climatic_niche_Ramirez_et_2016_fig1}
      \includegraphics<3>[width=\textwidth]{nicheRepro}
      \includegraphics<4>[width=\textwidth]{lanioFicedula}
      \end{center}
    \end{column}
  \end{columns}
  \tiny{\cite{Barbet-Massin2009,Barbet-Massin2012,Bonetti2014,RamirezAlbores2016}}
\end{frame}

% ========================================
\section{Dynamique de population}
% ========================================


% ----------------------------------------
\subsection{La démographie}
% ----------------------------------------

\begin{frame}{La démographie}
   \begin{columns}
    \begin{column}[c]{.5\textwidth}
      \begin{block}{Démographie}
        Etude des caractéristiques des populations, de
        leurs dynamiques, et des facteurs impliqués:
        \begin{itemize}[<+->]
        \item natalité
        \item mortalité
        \item immigration
        \item émigration
        \end{itemize}
      \end{block}
    \end{column}
    \begin{column}[c]{.5\textwidth}
      \begin{center}
        \includegraphics<1>[width=.7\textwidth]{naissance_cigogne}
        \includegraphics<2>[width=.7\textwidth]{mort}
        \includegraphics<3->[width=\textwidth]{metapopulations1}
      \end{center}
    \end{column}
  \end{columns}
\end{frame}

 \begin{frame}{Des cycles de vie et histoires de vie variés}
   \includegraphics[width=\textwidth]{cycle_vie}
 \end{frame}

\begin{frame}{Des compromis évolutifs}
  Combien dois je produire dedscendants chaque année ?
  \begin{itemize}[<+->]
  \item Raisonnement de \textbf{Lack}: la sélection va favoriser la taille de portée qui donne le +
    de descendants survivants\\~\\
   hypothèses
    \begin{itemize}
    \item les descendants ont la même taille (investissement énergie =)
    \item la \% de survie de chaque juvénile décroit avec la taille de la portée
    \end{itemize}
  \end{itemize}
\end{frame}


% ----------------------------------------
\subsection{Des compromis}
% ----------------------------------------



\begin{frame}{Des compromis évolutifs}
  Combien dois je produire de descendants chaque année ? \\~\\
\begin{columns}
    \begin{column}[c]{.5\textwidth}
      Test de l'hypothèse de lack
        \begin{itemize}[<+->]
        \item Ajout et soustraction de œufs à la taille modale de la couvée
de 7 œufs
        \item Les nichées manipulées ont produit moins de descendants
          que les nichées de la taille  modale 
        \item Respect de l'hypothèse ? OUI ou NON
        \item $\rightarrow$ \textbf{OUI}
        \end{itemize}
    \end{column}
    \begin{column}[c]{.5\textwidth}
      \begin{center}
      \includegraphics[width=.7\textwidth]{pie_nb_oeuf}
       \end{center}
    \end{column}
  \end{columns}
\end{frame}



 
\begin{frame}{Des compromis évolutifs}
  Combien dois je produire de descendants chaque année ? 
  \begin{columns}
    \begin{column}[c]{.5\textwidth}
      Mésange charbonnière
      \begin{itemize}[<+->]
      \item sur 40 ans en moyenne 8.5 oeufs par nichée
      \item la théorie suggère 12 oeufs par nichée
      
      \item Respect de l'hypothèse de Lack ? OUI ou NON
      \item $\rightarrow$ \textbf{NON}
   
      \end{itemize}
    \end{column}
    \begin{column}[c]{.5\textwidth}
      \begin{center}
        \includegraphics[width=.7\textwidth]{nid-mesange-charbonniere-oisillons-6}
      \end{center}
    \end{column}
  \end{columns}
\end{frame}


\begin{frame}{Des compromis évolutifs}
  Combien dois je produire de descendants chaque année ?\\~\\
  C'est pas si simple ! \\
  Hypothèses implicites de Lack:
  \begin{itemize}[<+->]
  \item pas de trade-off entre l’effort reproductif des parents une année donnée
    et sa survie ou reproduction dans les années futures
  \item \textbf{$\rightarrow$ FREQUEMMENT FAUX – évidences expérimentales chez le
      gobemouche à collier} \\ si on leur ajoute un oeuf, femelles pondent
    moins l’année suivante
  \item le seul effet de la taille de ponte est d’influer sur la proba de
    survie des poussins
  \item  \textbf{$\rightarrow$FREQUEMMENT FAUX – évidences expérimentales chez le
      gobemouche à collier }\\ correlation négative entre la taille de ponte d’une femelle et
    la taille de la ponte où elle a été élevée
  \item \textbf{La taille de ponte peut aussi affecter les performances reproductives
      de la descendance}
  \end{itemize}
\end{frame}



 \begin{frame}{Stratégie courte ou longue... un continuum}
   \includegraphics[width=\textwidth]{continuum_cout_long}
 \end{frame}




% ----------------------------------------
\subsection{Dynamique de population}
% ----------------------------------------



 
\begin{frame}{Dynamique de population}
 
  \begin{columns}
    \begin{column}[c]{.45\textwidth}
    
      \begin{itemize}[<+->]
      \item Si pas de contrainte croissance exponentielle \\~\\
        $N_{t+1} = rN_t$ \\
        où $r$ est le taux de croissance\\~\\
      \item mais croissance pas infinie \\
       $\rightarrow$ densité dépendence négative \\~\\
      $N_{t+1} = rN_t(1-\frac{N_t}{K})$ \\
      où $K$ est la capacité de charge pour l'espèce
       \end{itemize}
    \end{column}
    \begin{column}[c]{.55\textwidth}
      \begin{center}
        \includegraphics<1>[width=\textwidth]{croissance_exp}
        \includegraphics<2>[width=\textwidth]{croissance_logistique}
        
      \end{center}
    \end{column}
  \end{columns}
\end{frame}






% ----------------------------------------
\subsection{Effet Allee}
% ----------------------------------------



\begin{frame}{Effet Allee}
  \begin{columns}
    \begin{column}[c]{.45\textwidth}
      \begin{block}{Densité dépendence positive}
        \begin{itemize}[<+->]
        \item  Le taux de croissance peut diminuer à faible densité
          (ou à faible taille de pop)
        \item Les effets Allee sont généralement négligés dans les modèles de dynamique des
          pop (exemple pêcheries…), et pourtant…
        \end{itemize}
      \end{block}
    \end{column}
    \begin{column}[c]{.55\textwidth}
      \begin{center}
        \includegraphics<1->[width=\textwidth]{effet_allee_graph}
      \end{center}
    \end{column}
  \end{columns}
\end{frame}



 \begin{frame}{Effet Allee}
   \includegraphics<1>[width=.9\textwidth]{effet_allee_cause_1}
   \includegraphics<2>[width=.9\textwidth]{effet_allee_cause_2}
   \includegraphics<3>[width=.9\textwidth]{effet_allee_cause_3}
   \includegraphics<4>[width=.9\textwidth]{effet_allee_cause_4}
 \end{frame}


% ----------------------------------------
\subsection{Perte de biodiversité}
% ----------------------------------------


 \begin{frame}{Cause de perte de biodiversité}
L'"Evil quartet" de Jarod Diamond 1989
      \begin{center}
   \includegraphics[width=.8\textwidth]{evilquartet5}
      \end{center}
 \end{frame}


 \begin{frame}{Vortex d'extinction}
      \begin{center}
   \includegraphics[width=.8\textwidth]{AR-vortexDG}
      \end{center}
 \end{frame}





% ========================================
\section{Les interactions}
% ========================================

% ----------------------------------------
\subsection{Les interactions}
% ----------------------------------------
\begin{frame}{La classification des intéractions}
   \begin{columns}
    \begin{column}[c]{.5\textwidth}
      \includegraphics<1>[width=\textwidth]{interaction_1}
      \includegraphics<2>[width=\textwidth]{interaction_2}
      \includegraphics<3>[width=\textwidth]{interaction_3}
      \includegraphics<4>[width=\textwidth]{interaction_4}
      \includegraphics<5>[width=\textwidth]{interaction_5}
      \includegraphics<6>[width=\textwidth]{interaction_6}
    \end{column}
    \begin{column}[c]{.5\textwidth}
      \begin{center}
        \includegraphics<1>[width=0.5\textwidth]{commensalisme}
        \includegraphics<2>[width=\textwidth]{commensalism}
        \includegraphics<3>[width=\textwidth]{competition_mangeoire}
        \includegraphics<4>[width=\textwidth]{foret}
        \includegraphics<5-6>[width=.7\textwidth]{predation_parasite}
      \end{center}
    \end{column}
  \end{columns}
 \end{frame}


% ----------------------------------------
\subsection{Réseaux trophique}
% ----------------------------------------
\begin{frame}{Réseau trophique}
  \begin{center}
    \includegraphics<1>[width=\textwidth]{arbre-reseau-trophique-diurne1}
    \includegraphics<2>[width=.8\textwidth]{Representation-de-la-structure-du-reseau-trophique-du-Bamboung-en-2003-modelise-par}
  \end{center}
\end{frame}


\begin{frame}{Cycle de l'energie}
  \begin{center}
    \includegraphics<1>[width=\textwidth]{pyramide-des-c3a9nergies-dun-c3a9cosystc3a8me-aquatique}
    \includegraphics<2>[width=.75\textwidth]{energie_trophique}
     \includegraphics<3>[width=.75\textwidth]{chaine_trophique_marine}
  \end{center}
\end{frame}



% ----------------------------------------
\subsection{Proie-prédateur}
% ----------------------------------------

\begin{frame}{Modèle proie-prédateur}
  Compagnie de la baie d'Hudson au XIXe siècle
  \begin{center}
     \includegraphics[width=.75\textwidth]{lynx_lievre_dyn}
  \end{center}
\end{frame}


\begin{frame}{Modèle proie-prédateur}
  Modélisable simplement \\~\\

  \begin{columns}
   \begin{column}[c]{.5\textwidth}
      \begin{block}{Proies}
       $\frac{dH}{dt}=r_HH - b_HHP$
      \end{block}
   \end{column}
   \begin{column}[c]{.5\textwidth}
      \begin{block}{Prédateur}
        $\frac{dP}{dt}=r_Pb_HHP - b_PP$
      \end{block}
   \end{column}
 \end{columns}
  \vspace{1cm}
 où $r$ les coefficient d'accroisement et $b$ les taux de mortalité
\end{frame}


\begin{frame}{Modèle proie-prédateur}
 
 \begin{columns}
   \begin{column}[c]{.5\textwidth}
     \begin{center}
          \includegraphics<1->[width=\textwidth]{Lotka-Volterra_orbits_02}
  
     \end{center}
    \end{column}
    \begin{column}[c]{.5\textwidth}
      \begin{center}
         \includegraphics<2>[width=.9\textwidth]{Lotka-Volterra_orbits_01}
 
      \end{center}
   
    \end{column}
  \end{columns}
  
  \end{frame}


\begin{frame}{Modèle proie-prédateur}
 
  \begin{block}{Modèle très simplifié}
    Exercice intéressant mais des hypthèses de simplification :
    \begin{itemize}[<+->]
    \item pas de sexe et pas de couple pour la reproduction
    \item pas de classe d'age (efficacité et reproduciton)
    \item pas de problème de rencontre (effet Allee ?)
    \item une seule source de nouriture pour le lynx
    \item pas d'adaptation
    \end{itemize}
    
  \end{block}
  
  \end{frame}

\begin{frame}{Résilience et hystérésie}
  \begin{center}
    \includegraphics<1>[width=\textwidth]{Conceptual-representation-of-resistance-resilience-and-hysteresis-with-the-ball-incup}
  \end{center}
\end{frame}



\begin{frame}{Modèle proie-prédateur et la séléction naturelle}
  \begin{center}
    \includegraphics[width=.8\textwidth]{PCtyrannohunting}
  \end{center}
  \begin{itemize}[<+->]
  \item Deux petits dinosaures essaient d'echapper à un T.rex
    \begin{itemize}
    \item \textit{Je ne sais pas pourquoi on court, nous ne courerons jamais assez
        vite pour echapper au T.rex}
    \item \textit{Je ne cherche pas à courrir plus vite que le T.rex, je cherche
        seulement à courir plus vite que toi !} 
    \end{itemize}
  \end{itemize}
\end{frame}


\begin{frame}[plain]
  \begin{center}
    \begin{huge}
      Merci de votre attention...\\~\\
    \end{huge}
    \includegraphics[width=\textwidth]{18TB-REEF-facebookJumbo}
  \end{center}
\end{frame}

\begin{frame}[allowframebreaks]
  \begin{tiny}
    \frametitle{Réferences}
    \bibliographystyle{apalike}
    \bibliography{bib_files/biblio}
  \end{tiny}
\end{frame}

\end{document}


