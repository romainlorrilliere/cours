\documentclass[10pt]{beamer}
\usepackage[utf8]{inputenc}
%\usepackage[T1]{fontenc}
\usepackage[french]{babel}
% \usepackage{natbib}
%\usepackage{hyperref}
% \usepackage{animate}
%\usepackage{graphicx}
%\usepackage{color}

\pdfminorversion=5 
\pdfcompresslevel=9
\pdfobjcompresslevel=3

%% Theme
% \useinnertheme[shadow=true]{rounded}
% \useoutertheme{shadow}
% \usecolortheme{orchid}
% \usecolortheme{whale}%\usetheme{progressbar}
%\usetheme[]{Berlin}

\useoutertheme[footline=authorinstitutetitle]{miniframes}
\useoutertheme{smoothtree}
\useinnertheme[shadow=true]{rounded}
\usecolortheme{orchid}
\usecolortheme{whale}


\graphicspath{{./images/}} 

% \usecolortheme{progressbar}
% \usefonttheme{progressbar}
% \useinnertheme{progressbar}
% \useoutertheme{progressbar}
%% Logo
% \logo{\includegraphics[height=0.5cm]{./images/Logo.png}}
%% Fond de la page
% \setbeamercolor{background canvas}{bg=couleur}

\title[Suivi de populations d'oiseaux]{Suivi des populations d’animaux
  sauvages et dynamique de populations \\
  \textit{\footnotesize{Outils et méthodes adaptés aux enjeux, l'exemple des oiseaux des populations aux communautés}}}

\author{Romain Lorrilliere | \textit{romain.lorrilliere@mnhn.fr}}
\institute{ONIRIS : DIE Santé de la faune sauvage non captive (Janv 2021)}
\date{11 Janvier 2021}

\AtBeginSection[]{
  \begin{frame}
   \frametitle{\insertsectionhead}
    \scriptsize \tableofcontents[currentsection,hideothersubsections]
  \end{frame}
}



%\AtBeginSubsection[]{
%  \begin{frame}
%    \frametitle{\insertsubsectionhead}
%    \scriptsize \tableofcontents[sectionstyle=show/shaded,subsectionstyle=show/shaded/hide ]
%  \end{frame}
%}
%\addtobeamertemplate{footline}{\insertframenumber/\inserttotalframenumber}

\begin{document}
\maketitle



% \section{Sommaire}
%% =====================
\begin{frame}
   \frametitle{Sommaire}
   \tableofcontents[hideallsubsections]
 \end{frame}




% ===============================================
\section{Les oiseaux}
% ===============================================

\begin{frame}{Les oiseaux}
\begin{center}
	\includegraphics<1>[width=0.8\textwidth]{divOiseaux2_1}
\end{center}
 \end{frame}



% ----------------------------------------------------------
\subsection{Un groupe ancien} 
% ----------------------------------------------------------

\begin{frame}{Les dinosaures rescapés} 
        \begin{center}
        \includegraphics[width=\textwidth]{the-great-tree-of-life}     
      \end{center}
\end{frame}




\begin{frame}{Les dinosaures rescapés} 
  \begin{columns}
    \begin{column}[c]{0.55\textwidth}
    \begin{small}
      \begin{itemize}[<+->]
      \item Descendant de petits dinosaures \textit{Theropodes}\\
        \begin{center}
          \includegraphics<1->[width=.5\textwidth]{Jurassic-Park}   
        \end{center}
      \item Extinction suite à la crise Crétacé-Terciaire (volcanisme,
        météorite), \footnotesize{65,5 Ma env.}
      \item Importante radiation évolutive
      \end{itemize}
       \end{small}
    \end{column}
    \begin{column}[c]{0.45\textwidth}
      \begin{center}
        \includegraphics<1-2>[width=\textwidth]{bird_evo}     
        \includegraphics<3>[width=\textwidth]{tree_of_piaf}     
      \end{center}
    \end{column}
  \end{columns}
\end{frame}


\begin{frame}{Diversification : 10000 sp}
  \begin{center}
    \includegraphics[width=0.8\textwidth]{divOiseaux2_2} 
  \end{center}
\end{frame}

\begin{frame}{Diversification}
  \begin{columns}[c]
    \begin{column}[c]{0.35\textwidth}
      \begin{itemize}[<+->]
      \item Régime alimentaire
      \item Démographie
      \item Stratégie reproductive
      \item Cycle biologique nycthémère
      \item Habitat
      \item Region
      \item Mode déplacement
      \item Migration
      \end{itemize}
    \end{column}
    \begin{column}[c]{0.7\textwidth}
      \includegraphics[width=\textwidth]{divOiseaux2_2} \\
    \end{column}
  \end{columns}
\end{frame}


% ----------------------------------------------------------
\subsection{La migration} 
% ----------------------------------------------------------

\begin{frame}{La migration}
  \begin{center}
    La migration permet aux oiseaux d’échapper aux conditions rigoureuses de l'hiver sur leur site de reproduction\\
    \vspace{20pt}
    \includegraphics[width=.9\textwidth]{volMigration} \\
    \vspace{20pt}
    Phénomène très répandu chez les oiseaux : elle concerne 20\% des
    espèces et 40\% des espèces qui nichent en Europe et en Asie
  \end{center}
\end{frame}


\begin{frame}{La migration : Migrateurs de courte ou ...}
   \begin{columns}[c]
    \begin{column}[c]{0.5\textwidth}
      \begin{center}
       \includegraphics<1->[width=.55\textwidth]{migrationCourt} 
      \end{center}
    \end{column}
    \begin{column}[c]{0.5\textwidth}
      
        \end{column}
  \end{columns}
\end{frame}



\begin{frame}{La migration :  Migrateurs de courte ou ???}
    \begin{columns}[c]
    \begin{column}[c]{0.3\textwidth}
      \begin{itemize}[<+->]
      \item Les migrations déjà observées par Aristote (384-322
        avant notre ère)
      \item La migration courte distance
      \item Mais que la migration courte distance
      \end{itemize}
    \end{column}
    \begin{column}[c]{0.7\textwidth}
      \begin{center}
        \includegraphics<1>[width=\textwidth]{aristote_migration_1}
        \includegraphics<2>[width=\textwidth]{aristote_migration_2}
        \includegraphics<3>[width=\textwidth]{aristote_migration_3}
      \end{center}
    \end{column}
  \end{columns}
\end{frame}



\begin{frame}{La migration :  Migrateurs de courte ou hibernation ???}
   \begin{columns}[c]
    \begin{column}[c]{0.3\textwidth}
      \begin{itemize}
      \item Les migrations déjà observées par Aristote (384-322
        avant notre ère)
      \item La migration courte distance
      \item Mais que la migration courte distance
      \end{itemize}
    \end{column}
    \begin{column}[c]{0.7\textwidth}
      \begin{center}
         \includegraphics[width=\textwidth]{aristote_migration_3}
      \end{center}
    \end{column}
  \end{columns}
  \begin{center}
    Theorie de l'hibernation qui persiste jusquà la fin du 19ème
    siècle, mais remise en question dès le 16ème siecle et notamment par Buffon. 
  \end{center}
\end{frame}


\begin{frame}{La migration : Migrateurs de courte ou longue distance}
 \begin{columns}[c]
    \begin{column}[c]{0.5\textwidth}
      \begin{center}
       \includegraphics[width=.55\textwidth]{migrationCourt} 
      \end{center}
    \end{column}
    \begin{column}[c]{0.5\textwidth}
      \includegraphics[width=.55\textwidth]{migrationLong} 
        \end{column}
  \end{columns}
\end{frame}


\begin{frame}{La migration : Observation}
  \begin{columns}[c]
    \begin{column}[c]{0.6\textwidth}
      \begin{itemize}[<+->]
      \item Dès 1880 au US, observation des vols migratoires sur des points de passage
        important  \\
        En France ex : col d'Organbidexka (Pyrénées-Atlantiques)
      \item $\rightarrow$ comptage et date de passage
      \item  En 1899 et 1900 Hans Christian Cornelius Mortensen bague
        avec des bagues en aluminum des Etourneaux sansonnet pour
        connaitre leur destination hivernale
      \item Naissance du baguage des oiseaux
      \item $\rightarrow$ déterminiation grossière des routes migratoires
      \end{itemize}
    \end{column}
    \begin{column}[c]{0.4\textwidth}
      \begin{center}
        \includegraphics<1-2>[width=\textwidth]{organbidexka}
        \includegraphics<3>[width=\textwidth]{Mortensen}
        \includegraphics<4>[width=\textwidth]{controle_acrsch}
        \includegraphics<5>[width=\textwidth]{crbpodata_acrsch}
      \end{center}
    \end{column}
  \end{columns}
\end{frame}




\begin{frame}{La migration : Technologie embarquée}
  \begin{columns}[c]
    \begin{column}[c]{0.6\textwidth}
      \begin{itemize}[<+->]
      \item En 1978 création du system Argos \\
        \footnotesize{col. NASA, NOAA, CNES} \\
        \footnotesize{mais technologie lourde,\\ et $ < 3\% $  (max 5\%) masse oiseaux}
      \item En 1984 début de technologie embarquée aux US sur
        Pyguargue à tête blanche ($> 5 Kg$)\\
        Balise de $170g$
      \item ex: Aigle pomarin\\
        Mise en évidence des routes migratoires
      \item ex: Barge rousse\\
        Voyage incroyable de + de 10000 km en 5-7 jours sans stop
      \end{itemize} 
    \end{column}
    \begin{column}[c]{0.4\textwidth}
      \begin{center}
        \includegraphics<1>[width=\textwidth]{argos}
        \includegraphics<2>[width=\textwidth]{gps155g_pygargue_queue_blanche}
        \includegraphics<3>[width=.9\textwidth]{gps-aigle_pomarin}
        \includegraphics<4>[width=\textwidth]{barge_rousse_migration}
      \end{center}
    \end{column}
  \end{columns}
\end{frame}


\begin{frame}{La migration : Technologie embarquée}
  \begin{columns}[c]
    \begin{column}[c]{0.4\textwidth}
      \begin{itemize}[<+->]
      \item ex: Cigogne blanche
      \end{itemize} 
    \end{column}
    \begin{column}[c]{0.6\textwidth}
      \begin{center}
        \includegraphics[width=\textwidth]{movebank_cigogne_44}
      \end{center}
    \end{column}
  \end{columns}
\end{frame}

\begin{frame}{La migration: de nombreuses routes différentes}
  \begin{center}
    \includegraphics[width=.9\textwidth]{bird_migration} 
  \end{center}
\end{frame}


\begin{frame}{Les mouvements : Technologie embarquée}
  ex: Albatros à sourcils noirs de Kerguélen
  \begin{columns}[c]
    \begin{column}[c]{0.4\textwidth}
      \begin{center}
        \includegraphics[width=0.9\textwidth]{Argos_PTT_albatross}
      \end{center}
    \end{column}
    \begin{column}[c]{0.6\textwidth}
      \begin{center}
        \includegraphics[width=\textwidth]{albatross_journey_argos_kergelen}
      \end{center}
    \end{column}
  \end{columns}
\end{frame}




\begin{frame}{Dispersion}
  \begin{columns}[c]
    \begin{column}[c]{0.75\textwidth}
      ex: Les albatros hurleur de Kerguélen et Crozet
    \end{column}
    \begin{column}[c]{0.25\textwidth}
      \includegraphics[width=\textwidth]{Wandering-albatross-display} 
    \end{column}
  \end{columns}
  \begin{center}
    \includegraphics[width=0.8\textwidth]{wanderingAlbatrosMigration} 
  \end{center}
\end{frame}


% ----------------------------------------------------------
\subsection{Pourquoi les oiseaux ?} 
% ----------------------------------------------------------

\begin{frame}{Indicateur de biodiversité}
  \begin{center}
    \includegraphics[width=.8\textwidth]{indicateurs} 
  \end{center}
\tiny{\cite{Balmford2005,Donald2001,Donald2002,Doxa2010,Fisher2009,Gregory2005,Sekercioglu2004,Weibull2003}}
\end{frame}


\begin{frame}{Et surtout...}
  \begin{center}
    \includegraphics[width=.9\textwidth]{AEGCAU} 
  \end{center}
\end{frame}


\begin{frame}{Les observateurs}
  \begin{center}
    \includegraphics<1>[width=.9\textwidth]{birdwatcher} 
    \includegraphics<2>[width=.9\textwidth]{twitchers} 
    \end{center}
\end{frame}


\begin{frame}{Observations naturalistes}
   \begin{columns}[c]
    \begin{column}[c]{0.8\textwidth}
      Principalement des données opportunistes \\
      $\rightarrow$ sans protocole
    \end{column}
    \begin{column}[c]{0.20\textwidth}
      \includegraphics[width=\textwidth]{visionature_logo} 
    \end{column}
  \end{columns}
   \begin{center}
    \includegraphics[width=.8\textwidth]{faune_france_contribution_2021_01_09}
  \end{center}
  \footnotesize{source: http://https://www.faune-france.org/ (09/01/2021)}
\end{frame}


% =====================================================================
\section{Démarche scientifique}
% =====================================================================

\begin{frame}{La démarche scientifique}
  Méthode scientifique hypothético-déductive
  \begin{center}
      \includegraphics[width=.9\textwidth]{methode_scientifique_wikipedia}
    \end{center}
    Grande diversité dans les pratiques en fonction des
    domaines.
    \begin{tiny}
      (Wikipedia)
    \end{tiny}
\end{frame}

\begin{frame}{La démarche scientifique en écologie}
  \begin{itemize}
  \item En écologie, il est en général impossible d’étudier une entité
    dans sa totalité \\
    \begin{footnotesize}
    Temps, accessibilité, etc…      
    \end{footnotesize}
  \item Travail sur une sous-partie de l’objet d’étude (=échantillon)
    \begin{itemize}
    \item Problème : on veut en général pouvoir extrapoler les
      observations de l’échantillon à l’ensemble de l’objet d’étude
    \item L’échantillon doit refléter la composition, la complexité, la variabilité de l’objet d’étude
    \end{itemize}
  \item $\rightarrow$ un plan d’échantillonnage est nécessaire
  \end{itemize}
\end{frame}


\begin{frame}{Rôle des statistiques dans les plans d’échantillonnage}
  Différence entre probabilités et statistiques
  \begin{columns}[c]
    \begin{column}[c]{0.5\textwidth}
      \begin{block}<2->{Probabilité}
        \begin{center}
          \includegraphics[width=.3\textwidth]{prob_stat_prob}
        \end{center}
        \begin{small}
          La théorie des probabilités permet de faire des inférences sur un échantillon, connaissant la population dont il est issu
        \end{small}
      \end{block}
    \end{column}
    \begin{column}[c]{0.5\textwidth}
      \begin{block}<3->{Statistique}
        \begin{center}
          \includegraphics[width=.3\textwidth]{prob_stat_stat}
        \end{center}
        \begin{small}
          Les statistiques permettent d’inférer les caractéristiques
          d’une population à partir de la connaissance d’un échantillon
        \end{small}
      \end{block}
    \end{column}
  \end{columns}
  \begin{alertblock}<4->{}
    \begin{center}
      $\rightarrow$ Rôle \textbf{crucial des statistiques} pour la construction et l’interprétation des plans d’échantillonnage   
    \end{center}
  \end{alertblock}
\end{frame}


\begin{frame}{Définition du plan d'échantillonnage}
  \begin{columns}[c]
    \begin{column}[c]{0.7\textwidth}
      \begin{small}
        \begin{enumerate}[<+->]
        \item Définir le (ou les) objectif(s) \\
          \only<1> {
            \begin{footnotesize}
              écologie théorique, conservation, indicateurs
            \end{footnotesize}}
        \item Définir la population statistique à échantillonner\\
          \only<2> {
            elle doit être bien délimitée dans l’espace et dans le temps \\
            \begin{footnotesize}
              groupe taxonomique, unité d'échantillonnage, population
            \end{footnotesize}}
        \item Définir les variables à mesurer\\
          \only<3> {
            \begin{footnotesize}
              mesure sur la population statistique (abondance, masse, taille
              d'aile, age) et sur les variables
              explicatives (carte, SIG...)
            \end{footnotesize}}
        \item Identifier les sources de variabilité et d'erreur\\
          \only<4> {
            \begin{footnotesize}
              effet observateur, d'apprentissage
            \end{footnotesize}}
        \item Définir le mode de répétition des mesures\\
          \only<5> {
            \begin{footnotesize}
              spatiale ou temporelle
            \end{footnotesize}}
        \item Positionner les unités d'échantillonnage 
          \only<6> {
            \begin{footnotesize}
              tirage aléatoire ou échantillonnage structuré ou systématique
            \end{footnotesize}}
        \end{enumerate}
      \end{small}
    \end{column}
    \begin{column}[c]{0.3\textwidth}
      \includegraphics<1>[width=\textwidth]{objectif}
      \includegraphics<2>[width=\textwidth]{pop_stat}
      \includegraphics<3>[width=.7\textwidth]{variable}
      \includegraphics<4>[width=\textwidth]{bruit_biais}
      \includegraphics<5->[width=\textwidth]{repetition}
    \end{column}
  \end{columns}
\end{frame}


\begin{frame}{Plan d'échantillonnage et faune sauvage}
  Ethique de la manipulation de la faune sauvage.
  Bien être des animaux est compromis par
  \begin{itemize}
  \item le stress 
  \item les risques de blessure et de mortalité
  \end{itemize}

  \begin{alertblock}<2->{Pourquoi est-ce un problème ?}
    \begin{itemize}
    \item Risque pour l’animal si stress maintenu plus de quelques
      minutes 
    \item Risque pour la science: l’activation physiologique d’une réponse au stress interfère avec les objets mêmes de la recherche
    \end{itemize}
    \begin{center}
      \textbf{Happy animal make good science}
    \end{center}
    \begin{tiny}
      (Poole T., 1997)
    \end{tiny}
    
  \end{alertblock}
\end{frame}


\begin{frame}{Plan d'échantillonnage: les 3Rs}
  \begin{block}{les 3Rs}
  \begin{itemize}[<+->]
  \item \textbf{Remplacer}\\
    Ne pas embêter d'animaux
  \item \textbf{Réduire}\\
    En embêter moins
  \item \textbf{Raffiner}\\
    Embêter le moins possible
  \end{itemize}
  \end{block}
  \begin{tiny}
  Principles of Humane Experimental Techniques (Russell and Burch 1959)
  \end{tiny}
\end{frame}



\begin{frame}{Plan d'échantillonnage: Remplacer !}
    \begin{columns}[c]
    \begin{column}[c]{0.7\textwidth}
     \begin{itemize}[<+->]
  \item bibliographie
  \item méthode sans capture
    \begin{itemize}[<+->]
    \item piege photo
    \item enregistrement sonore
    \item observation
    \end{itemize}
  \end{itemize}
    \end{column}
    \begin{column}[c]{0.3\textwidth}
      \includegraphics<1>[width=\textwidth]{presse}
      \includegraphics<3>[width=.8\textwidth]{piege_photo}
      \includegraphics<4>[width=.8\textwidth]{piege_son}
      \includegraphics<5>[width=\textwidth]{jumelles}
    \end{column}
  \end{columns}
\end{frame}


% ----------------------------------------------------------
\section{Sciences participatives} 
% ----------------------------------------------------------
\begin{frame}{Sciences participatives}
  \begin{columns}[c]
    \begin{column}[c]{0.6\textwidth}
      \textbf{Vigie-Nature} un programme du \\\textbf{Muséum National d'Histoire
      Naturelle}\\
      \begin{tiny}
        http://www.vigienature.fr/ 
      \end{tiny}
    \end{column}
    \begin{column}[c]{0.3\textwidth}
      \includegraphics[width=\textwidth]{vigie_nature}
    \end{column}
    \begin{column}[c]{0.1\textwidth}
      \includegraphics[width=\textwidth]{mnhn}
    \end{column}
  \end{columns}
  \begin{center}
      \includegraphics<1>[width=.7\textwidth]{vigie-nature_protocole}
  \end{center}
  \begin{itemize}[<+->]
  \item  Suivis à large échelle et à long terme des espèces communes
    permettent de répondre à des questions essentielles sur la
    biodiversité ordinaire :\\
    \begin{itemize}[<+->]
    \item Quel sont les transformations quantitatives de notre faune
      commune ?
    \item Comment les espèces communes répondent aux pressions d'origine
      anthropique ? 
    \item Quel sont les effets du changement climatique ? 
    \end{itemize}
  \end{itemize}
\end{frame}

\begin{frame}{Protocole STOC (Breeding Bird Survey)}
  \begin{center}
    STOC: \textcolor{red}{S}uivi \textcolor{red}{T}emporel des \textcolor{red}{O}iseaux \textcolor{red}{C}ommuns
  \end{center}
  \begin{columns}[c]
    \begin{column}[c]{0.5\textwidth}
      \begin{center}
        2300 carrés suivis au moins une fois depuis 2001 \\
        \includegraphics[width=.8\textwidth]{carreSTOC_map_simple_}
      \end{center}
    \end{column}
    \begin{column}[c]{0.5\textwidth}
      \begin{small}
        \begin{itemize}
        \item  Sites choisis aléatoirement
        \item 10 pts d'écoute de 5mn par site
        \item 2 passages pas an
        \item Description standardisée de l'hab.
        \end{itemize}
      \end{small}
      \begin{center}
        \includegraphics[width=.7\textwidth]{STOC_points}
      \end{center}
    \end{column}
  \end{columns}
  \begin{center}
    175 espèces suivies
  \end{center}
\end{frame}


\begin{frame}{Protocole STOC (Breeding Bird Survey)}
  \begin{center}
    \includegraphics[width=.5\textwidth]{carreSTOC_}
  \end{center}
\end{frame}


\begin{frame}{STOC: Bruant jaune \textit{Emberiza citrinella}}
  \begin{columns}[c]
    \begin{column}[c]{0.5\textwidth}
      \includegraphics[width=.5\textwidth]{bruantjaune}
      \begin{itemize}
      \item  Tendance :
        \begin{itemize}
        \item $-50\%$ depuis 2001, diminution
        \end{itemize}
           \end{itemize}
    \end{column}
    \begin{column}[c]{0.5\textwidth}
      \begin{center}
        \includegraphics[width=.9\textwidth]{embcit2017}
      \end{center}
    \end{column}
  \end{columns}
\end{frame}


\begin{frame}{STOC: Pigeon ramier \textit{Columba palumbus}}
  \begin{columns}[c]
    \begin{column}[c]{0.5\textwidth}
      \includegraphics[width=.5\textwidth]{pigeonRamier}
      \begin{itemize}[<+->]
      \item  Tendance :
        \begin{itemize}
        \item $+78\%$ en 18 ans, augmentation
        \end{itemize}
      \item Distribution géographique 
      \end{itemize}
    \end{column}
    \begin{column}[c]{0.5\textwidth}
      \begin{center}
        \includegraphics<1-3>[width=.9\textwidth]{colpal_}
        \includegraphics<4->[width=\textwidth]{distributionPigeonRamier} 
      \end{center}
    \end{column}
  \end{columns}
\end{frame}


\begin{frame}{STOC: Indicateur groupe de specialisation}
  \begin{center}
     \includegraphics<1>[width=\textwidth]{STOC_indicateurgroupe}
  \end{center}
\end{frame}


\begin{frame}{STOC: Effet du changement climatique}
  \begin{center}
      \includegraphics<1>[width=.8\textwidth]{CTI}
  \includegraphics<2>[width=.9\textwidth]{CTIevolution}
  \end{center}
\end{frame}



% ----------------------------------------------------------
\section{Le suivi individuel} 
% ----------------------------------------------------------



\begin{frame}{Le suivi individuel}
  Pourquoi un suivi individuel ?
 \begin{columns}[c]
    \begin{column}[c]{0.7\textwidth}
     \begin{itemize}[<+->]
  \item Accès aux paramètres démographiques
    \begin{itemize}[<+->]
    \item Taux de survie
    \item Taux d'émigration
    \item Taux d'immigration
    \item Taux de recrutement
    \item Age et sexe ratio
    \end{itemize}
  \end{itemize}
  
    \end{column}
    \begin{column}[c]{0.3\textwidth}
      \begin{center}
        \includegraphics[width=.9\textwidth]{cycle_demographique}
      \end{center}
    \end{column}
  \end{columns}
\end{frame}


\begin{frame}{Le suivis individuel : le baguage}
  \begin{columns}[c]
    \begin{column}[c]{0.6\textwidth}
      \textbf{CRBPO} une une structure du \\\textbf{Muséum National d'Histoire
      Naturelle}\\
      \begin{tiny}
        https://crbpo.mnhn.fr/
      \end{tiny}
    \end{column}
    \begin{column}[c]{0.3\textwidth}
      \includegraphics[width=\textwidth]{crbpo_logo}
    \end{column}
    \begin{column}[c]{0.1\textwidth}
      \includegraphics[width=\textwidth]{mnhn}
    \end{column}
  \end{columns}
  \begin{columns}[c]
    \begin{column}[c]{0.6\textwidth}
     \begin{itemize}[<+->]
  \item Baguage pour la migration dès 1900
  \item Principe: pause d'une bague très légère (un 500ème de la masse
    de l'oiseau) avec un numéro unique
  \item en France à partir de 1911
  \item Aujourd'hui, aussi utilisé pour l'estimation des paramètres démographiques
  \end{itemize}
    \end{column}
    \begin{column}[c]{0.4\textwidth}
      \begin{center}
        \includegraphics[width=.9\textwidth]{bagues}
      \end{center}
    \end{column}
  \end{columns}

\end{frame}


\begin{frame}{Le baguage exemple simple}
  Programme national de baguage de suivi de la migration automnale
  avec pour but de documenter les variations de phénologie migratoire
  des passereaux entre individus, dans l’espace et dans le temps
  \begin{columns}[c]
    \begin{column}[c]{0.4\textwidth}
      \begin{small}
        \begin{itemize}[<+->]
        \item Les stations du programme PHENO
        \item Date de passage des Phragmites des joncs plus tardive si reproduction plus nordique
        \item Les jeunes Phragmites aquatiques sont plus tardifs
        \end{itemize}
      \end{small}
    \end{column}
    \begin{column}[c]{0.6\textwidth}
      \begin{center}
        \includegraphics<1>[width=\textwidth]{carte_stations_pheno_2013_2014}
        \includegraphics<2>[width=.3\textwidth]{acrsch}
        \includegraphics<2>[width=.8\textwidth]{pheno_date_passage}
        \includegraphics<3>[width=.3\textwidth]{acrola}
        \includegraphics<3>[width=.8\textwidth]{pheno_sex_acrola}
      \end{center}
    \end{column}
  \end{columns}
\end{frame}

\begin{frame}{Capture-Marquage-Recapture (CMR)}
  \begin{block}{Définition}
    Les modèles CMR, sont des modèles d'inférence mathématique
    permettant d'estimer la taille d'une population en estimant
    également la probabilité de capture des individus. 
  \end{block}
\end{frame}

\begin{frame}{Capture-Marquage-Recapture (CMR)}
  \begin{block}{Principe}
    \textbf{Capture} et \textbf{marquage} d'une partie des individus
    d'une population. La part des \textbf{recaptures} lors d'une seconde
    session permettra d'estimer la taille de la population. 
  \end{block}
  \begin{center}
     \includegraphics[width=.7\textwidth]{Cmr_exemple}
  \end{center}
  \begin{tiny}
    Par EDYP Vandutiours — Travail personnel, CC BY-SA 4.0, https://commons.wikimedia.org/w/index.php?curid=44095834
  \end{tiny}
\end{frame}



\begin{frame}{Capture-Marquage-Recapture (CMR)}
  \begin{block}{Hypothèses}
    \begin{itemize}[<+->]
    \item \textbf{Population close}\\
      \begin{footnotesize}
        pas d'entrée (immigration, naissance) ou de sortie
        (émigration, mort) d'individus entre les deux sessions 
      \end{footnotesize}
    \item \textbf{Probabilité de capture équivalente pour tous les individus}\\
      \begin{footnotesize}
        entre les individus et entre les sessions
      \end{footnotesize}
    \item \textbf{Marquage dois être durable}\\
      \begin{footnotesize}
        les individus marqués ne doivent pas perdre leur marque
      \end{footnotesize}
    \item \textbf{Les individus marqués se mélangent à la population}\\
      \begin{footnotesize}
        et ont la même probabilité de capture que les individus non marqués
      \end{footnotesize}
    \end{itemize}
  \end{block}
\end{frame}


\begin{frame}{Capture-Marquage-Recapture (CMR)}
  \begin{block}{Importance des recaptures}
    Pour augmenter les probabilités de \emph{recatpure} il exsite de nombreuses
    méthodes de marquage qui permettent un contrôle à distance des
    individus    
  \end{block}
  \begin{center}
    \includegraphics[width=.6\textwidth]{marquage}
  \end{center}
  \only<2>{Modèles autorisant une différence de probabilité de capture entre
    la première capture et les recaptures}
\end{frame}



\begin{frame}{Construction d'un modèle CMR}
  \begin{center}
    \includegraphics<1>[width=\textwidth]{cmr1}   
    \includegraphics<2>[width=\textwidth]{cmr2}
    \includegraphics<3>[width=\textwidth]{cmr3}
    \includegraphics<4>[width=\textwidth]{cmr4}
  \end{center}  
\end{frame}

\begin{frame}{Analyse matriciel d'un CMR}
  \begin{center}
    \includegraphics<1>[width=.5\textwidth]{math}   
  \end{center}  
\end{frame}

\begin{frame}{Analyse matriciel d'un CMR}
  \begin{center}
    \includegraphics<1>[width=.8\textwidth]{matrice1}   
    \includegraphics<2>[width=.8\textwidth]{matrice2}
    \includegraphics<3>[width=.8\textwidth]{matrice3}
    \includegraphics<4>[width=.8\textwidth]{matrice4}
    \includegraphics<5>[width=.8\textwidth]{matrice10}
    \includegraphics<6>[width=.8\textwidth]{matrice6}
    \includegraphics<7>[width=.8\textwidth]{matrice7}
    \includegraphics<8>[width=.8\textwidth]{matrice11}
  \end{center}  
\end{frame}



\section{Exemple d'étude de conservation}

\begin{frame}{Le Bruant ortolan}
  \begin{columns}[c]
    \begin{column}[c]{0.5\textwidth}
      \begin{itemize}[<+->]
      \item Un petit passereau de la famille des bruants, gros comme un
        moineau 
      \item Principalement granivore, et insectivore lors de l'élevage
        des jeunes
      \item Se reproduit en Europe et en Russie
      \item Hiverne en Afrique sub-saharienne
      \end{itemize}
    \end{column}
    \begin{column}[c]{0.5\textwidth}
      \begin{center}
        \includegraphics[width=.5\textwidth]{ortolan_male}
        \includegraphics[width=.7\textwidth]{ortolan_carte}
      \end{center}
    \end{column}
  \end{columns}
\end{frame}


\begin{frame}{Le Bruant ortolan : les menaces}
    \begin{columns}[c]
    \begin{column}[c]{0.5\textwidth}
      \begin{itemize}[<+->]
      \item Entre 3 et 7 millions de couples en Europe
        \begin{itemize}
        \item Principalement en Russie et Turquie (90 \%) \\
          Tendance -10\% depuis 2000
        \item France 6000 couples dans le centre et le sud-est
        \item Norvège, Suède, ouest de la Finland : env. 10000
          couples \\
          Tendance -30\% depuis 2000 
       \end{itemize}
     \end{itemize}
    \end{column}
    \begin{column}[c]{0.5\textwidth}
      \begin{center}
          \includegraphics[width=.7\textwidth]{ortolan_carte}
      \end{center}
    \end{column}
  \end{columns}
\end{frame}


\begin{frame}{Le Bruant ortolan : les menaces}
    \begin{columns}[c]
    \begin{column}[c]{0.5\textwidth}
      \begin{itemize}[<+->]
      \item Perte d'habitat
      \item Dinimution de l'abondance des proies
      \item Pratique traditionnelle dans les Landes
        \begin{itemize}
        \item Braconnage: capture à la matole (jusqu’en 2018)\\
          1500 braconniers \\
          $\times$ 30 matoles \\
          $\times$ 5 appelants \\
          $=$ peut être 30000 captures par an en automne
        \item Engraissement au grain
        \item Tué à l’Armagnac
        \item Plat culturel : Ortolan cuit à l’Armagnac
       \end{itemize}
     \end{itemize}
    \end{column}
    \begin{column}[c]{0.5\textwidth}
      \begin{center}
       \includegraphics<3->[width=.7\textwidth]{ortolan_braconnage}
      \end{center}
    \end{column}
  \end{columns}
\end{frame}


\begin{frame}{Le Bruant ortolan : problématique}
  \begin{columns}[c]
    \begin{column}[c]{0.5\textwidth}
      \begin{itemize}[<+->]
      \item Espèce protégée mais besoin de mieux comprende les routes
        de migration 
      \item D'où viennent les oiseaux qui passent dans le sud-ouest de
        la France
      \end{itemize}
    \end{column}
    \begin{column}[c]{0.5\textwidth}
      \begin{center}
        \includegraphics[width=.5\textwidth]{ortolan_male}
        \includegraphics[width=.7\textwidth]{ortolan_carte_2}
      \end{center}
    \end{column}
  \end{columns}
\end{frame}

\begin{frame}{Le Bruant ortolan : problématique}
  \begin{center}
    \includegraphics[width=.5\textwidth]{migration_route_europe}
  \end{center}
\end{frame}

\begin{frame}{Le Bruant ortolan :le plan d'échantillonnage}
   \begin{columns}[c]
    \begin{column}[c]{0.5\textwidth}
      \begin{itemize}[<+->]
      \item Echantillonnage de nombreuses populations en Europe
      \item Le long d'un gradient Est-Ouest
      \end{itemize}
    \end{column}
    \begin{column}[c]{0.5\textwidth}
     \begin{center}
    \includegraphics[width=\textwidth]{ortolan_echantillonnage}
  \end{center}
    \end{column}
  \end{columns}
 \end{frame}


\begin{frame}{Le Bruant ortolan : méthode de suivis de la migration}
 \begin{columns}[c]
    \begin{column}[c]{0.5\textwidth}
      \begin{itemize}[<+->]
      \item Le Bruant ortolan pèse env. 20g
      \item GPS ARGOS beaucoup trop lourd
      \item Géo-localisateur (GLS) 0,5g \\
        (2,5\% de la masse de l'oiseau)
      \end{itemize}
    \end{column}
    \begin{column}[c]{0.5\textwidth}
      \begin{center}
        \includegraphics<3>[width=.5\textwidth]{gls}
      \end{center}
    \end{column}
  \end{columns}
\end{frame}  


\begin{frame}{Les GLS : technologie}
\begin{columns}[c]
    \begin{column}[c]{0.5\textwidth}
      \begin{itemize}[<+->]
      \item Ne calculent pas de localisation
      \item Ne transmettent pas de données
      \item Mesurent l'intensité lumineuse toutes les minutes
      \item Doivent être récupérés pour obtenir les données    
      \end{itemize}
    \end{column}
    \begin{column}[c]{0.5\textwidth}
      \begin{center}
        \includegraphics<1-2>[width=.5\textwidth]{gls}
        \includegraphics<3->[width=.5\textwidth]{gls_lumiere}
      \end{center}
    \end{column}
  \end{columns}
\end{frame}


\begin{frame}{Les GLS : estimation des localisations}
  \begin{columns}[c]
    \begin{column}[c]{0.4\textwidth}
      \begin{itemize}[<+->]
      \item à partir de l'heure du levé du jour
      \item et de la longeur du jour
      \item fortes incertitudes proche de l'équateur et de l'équinoxe
      \end{itemize}
    \end{column}
    \begin{column}[c]{0.6\textwidth}
      \begin{center}
        \includegraphics[width=\textwidth]{gls_localisation}
      \end{center}
    \end{column}
  \end{columns}
\end{frame}

\begin{frame}{Terrain : Europe et Russie occidentale}
  \begin{center}
    \includegraphics[width=\textwidth]{ortolan_terrain_1}
  \end{center} 
\end{frame}

\begin{frame}{Terrain : Recherche des ortolans}
  \begin{center}
    \includegraphics[width=\textwidth]{ortolan_terrain_2}
  \end{center}
  \begin{columns}[c]
    \begin{column}[c]{0.5\textwidth}
     \begin{center}
        Femelle
      \end{center} 
    \end{column}
    \begin{column}[c]{0.5\textwidth}
      \begin{center}
        Mâle
      \end{center}
    \end{column}
  \end{columns}
\end{frame}


\begin{frame}{Terrain : La capture}
  \begin{center}
    \includegraphics<1>[width=\textwidth]{ortolan_terrain_3}
    \includegraphics<2>[width=\textwidth]{ortolan_terrain_4}
  \end{center}
  Diffusion de chant car les mâles sont très territoriaux
\end{frame}


\begin{frame}{Terrain : Pose du GLS}
  \begin{columns}[c]
    \begin{column}[c]{0.7\textwidth}
      \begin{itemize}
      \item Méthode du \emph{leg-loop}
        \begin{itemize}
        \item fixation sur le bas du dos par un lien qui fait une boucle
          derrière chaque pattes
        \end{itemize}
      \item Ajout d'une combinaison de bague couleur pour reconnaitre
        l'oiseau à distance 
      \end{itemize}
    \end{column}
    \begin{column}[c]{0.3\textwidth}
      \begin{center}
       \includegraphics[width=0.8\textwidth]{leg-loop_Thaxter_et_al_2014}
      \end{center} 
    \end{column}
  \end{columns}
\end{frame}

\begin{frame}{Terrain : Pose du GLS}
  \begin{center}
    \includegraphics<1>[width=\textwidth]{ortolan_terrain_5}
    \includegraphics<2>[width=0.4\textwidth]{ortolan_terrain_6}
    \includegraphics<3>[width=0.8\textwidth]{ortolan_terrain_7}
  \end{center} 
\end{frame}


\begin{frame}{Terrain : l'année suivante}
  \begin{center}
    \includegraphics[width=0.5\textwidth]{ortolan_terrain_8}
  \end{center} 
\end{frame}

\begin{frame}{Terrain : Recapture (Force et honneur ;-) )}
  \begin{center}
    \includegraphics<1>[width=\textwidth]{ortolan_terrain_9}
    \includegraphics<2>[width=\textwidth]{ortolan_terrain_10}
    \includegraphics<3>[width=0.5\textwidth]{ortolan_terrain_11}
  \end{center} 
\end{frame}

\begin{frame}{Terrain : La Recapture}
  \begin{center}
    \includegraphics[width=0.7\textwidth]{ortolan_terrain_12}
  \end{center}
  \begin{alertblock}{VICTOIRE !!!}
    Dans ce GLS il y a les données permettant de retracer la route de cet ortolan pendant 1 année !!!! 
  \end{alertblock}
\end{frame}



\begin{frame}{Les tracées}
 \begin{columns}[c]
    \begin{column}[c]{0.5\textwidth}
      \begin{itemize}[<+->]
      \item 41 GLS retrouvés qui ont fonctionnés
      \item 2 routes
        \begin{itemize}
        \item Une à l’ouest
        \item Une à l’est
        \end{itemize}
      \item Seuls les oiseaux provenant de Scandinavie occidentale et
        d’Allemagne passent par le Sud-Ouest de la France 
      \end{itemize}
    \end{column}
    \begin{column}[c]{0.5\textwidth}
      \begin{center}
        \includegraphics<1>[width=0.8\textwidth]{ortolan_gls_track_1}
        \includegraphics<2-4>[width=0.8\textwidth]{ortolan_gls_track_2}
        \includegraphics<5>[width=0.8\textwidth]{ortolan_gls_track_3}
      \end{center} 
    \end{column}
  \end{columns}  
\end{frame}

\begin{frame}{La mue}
  \begin{columns}[c]
    \begin{column}[c]{0.5\textwidth}
      \begin{itemize}[<+->]
      \item Les plumes sont fragiles et doivent être régulièrement
        remplacées
      \item Phénomène et sa phénologie très étudiés
      \item Chez les passereaux : mue annuelle
       \end{itemize}
    \end{column}
    \begin{column}[c]{0.5\textwidth}
      \begin{center}
        \includegraphics<1>[width=\textwidth]{mue_0_moineau}
        \includegraphics<2->[width=\textwidth]{mue_1}
      \end{center} 
    \end{column}
  \end{columns}  
\end{frame}


\begin{frame}{La mue : chez le Bruant ortolan}
  \begin{columns}[c]
    \begin{column}[c]{0.5\textwidth}
      \begin{itemize}[<+->]
      \item Bien connue
      \item Complète après la reproduction
      \item Partielle (petite plume du corps) en hiver avant le migration de retour
      \end{itemize}
    \end{column}
    \begin{column}[c]{0.5\textwidth}
      \begin{center}
           \includegraphics<1-3>[width=\textwidth]{mue_2}
        \includegraphics<4>[width=0.9\textwidth]{mue_3_ortolan}
      \end{center} 
    \end{column}
  \end{columns}  
\end{frame}


\begin{frame}{Isotope stable : le deutérium}
  Le deutérium, symbolisé 2H ou D, est un isotope naturel stable de
  l'hydrogène, et son abondance sur terre varie dans l'espace
  \begin{center}
    \includegraphics[width=0.8\textwidth]{deuterium}
  \end{center}
  \begin{block}<2>{}
    \begin{center}
      La composition des plumes dépend du lieu où elles poussent
    \end{center}
  \end{block}
\end{frame}

\begin{frame}{Où poussent les plumes l'hiver}
  \begin{center}
    \includegraphics<1>[width=0.8\textwidth]{ortolan_koweit_1}
    \includegraphics<2>[width=0.8\textwidth]{ortolan_koweit_2}
    \includegraphics<3>[width=0.8\textwidth]{ortolan_koweit_3}
    \includegraphics<4>[width=0.6\textwidth]{ortolan_koweit_4}
  \end{center}
    \only<4>{Prélèvement d'une plume scapulaire, muée en Afrique}
\end{frame}


\begin{frame}{Resultat du dosage}
  \begin{columns}[c]
    \begin{column}[c]{0.5\textwidth}
       \begin{block}<1->{à l'ouest}
         238 individus
         \includegraphics<2>[width=.9\textwidth]{ortolan_deuterium_ouest_1}
         \includegraphics<3->[width=.9\textwidth]{ortolan_deuterium_ouest_2}
       \end{block}
    \end{column}
    \begin{column}[c]{0.5\textwidth}
       \begin{block}<3->{à l'est}
         297 individus
         \includegraphics<4>[width=.9\textwidth]{ortolan_deuterium_est_1}
         \includegraphics<5>[width=.9\textwidth]{ortolan_deuterium_est_2}
       \end{block}
    \end{column}
  \end{columns}  
\end{frame}

\begin{frame}{La migration du Bruant ortolan}
   \begin{center}
      \includegraphics[width=\textwidth]{ortolan_resultats_migration}
  \end{center}
 \begin{block}<2->{}
    Les ortolans qui passent par la France proviennent de la
    population la plus petite et en fort déclin de Scandinavie
  \end{block}
  \begin{alertblock}<3>{}
    \begin{center}
      Le braconnage français fait payer un lourd tribu à cette
      population et sa probabilité d’extinction diminuera beaucoup
      si on y met fin.
    \end{center}
  \end{alertblock}
\end{frame}


\begin{frame}{Effet du braconnage}
  \begin{columns}[c]
    \begin{column}[c]{0.5\textwidth}
      \begin{itemize}
      \item Modèle de dynamique de population calibré sur les données
        démographiques Norvégiennes
      \item Fort effet du braconnage (rectangle rouge)
      \end{itemize}
    \end{column}
    \begin{column}[c]{0.5\textwidth}
       \begin{center}
      \includegraphics[width=.7\textwidth]{ortolan_result_sensibility}
  \end{center}
    \end{column}
  \end{columns}
    \begin{center}
      \includegraphics<2>[width=.7\textwidth]{ortolan_publi}
  \end{center}
\end{frame}


\begin{frame}[plain]
  \begin{center}
    \includegraphics[width=\textwidth]{ortolan_envol}\\
    \vspace{0.5cm}
    \begin{Large}
      Merci pour votre attention...      
    \end{Large}
  \end{center}
\end{frame}


\begin{frame}[allowframebreaks]
  \begin{tiny}
    \frametitle{Réferences}
    \bibliographystyle{apalike}
    \bibliography{bib_files/biblio}
  \end{tiny}
\end{frame}

\end{document}
