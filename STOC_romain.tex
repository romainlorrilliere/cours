\documentclass[]{beamer}
\usepackage[utf8]{inputenc}
\usepackage[T1]{fontenc}
\usepackage[english]{babel}
\usepackage{natbib}
\usepackage{hyperref}
%\usepackage{animate}
\usepackage{graphicx}
\usepackage{color}

%% Theme
\mode<presentation>
%\usetheme[]{Berlin}
\useoutertheme[footline=authorinstitutetitle]{miniframes}
\useoutertheme{smoothtree}
\useinnertheme[shadow=true]{rounded}
\usecolortheme{orchid}
\usecolortheme{whale}


\graphicspath{{./images/}} 


% \usecolortheme{progressbar}
% \usefonttheme{progressbar}
% \useinnertheme{progressbar}
% \useoutertheme{progressbar}
%% Logo
%\logo{\includegraphics[height=0.5cm]{Logo.png}}
%% Fond de la page
% \setbeamercolor{background canvas}{bg=couleur}

\title[STOC: DB et tendances]{STOC EPS : la base de données et le
  calcul des tendances}

\author{Romain Lorrillière | \textit{romain.lorrilliere@mnhn.fr}}
%\institute{Paris SUD : M1 EBE Module DYNA (Dec 2019)}
\date{14 Avril 2021}

\AtBeginSection[]{
  \begin{frame}
    \frametitle{\insertsectionhead}
    \scriptsize \tableofcontents[currentsection,hideothersubsections]
  \end{frame}
}

\addtobeamertemplate{footline}{\insertframenumber/\inserttotalframenumber}

\begin{document}
\maketitle



%\section{Sommaire}
%% =====================
%\begin{frame}
%  \frametitle{Sommaire}
%  \tableofcontents[hideallsubsections]
%\end{frame}


\begin{frame}{Plan}
  \tableofcontents[pausesections]
\end{frame}




%=======================================================
\section{La base de données}
% ===============================================


\begin{frame} {Quelques dates clefs}
  \begin{itemize}[<+->]
  \item \textbf{1989}
    \begin{itemize}
    \item début du STOC,
    \item les observateurs choisissent leur carré
    \item mais pas de base de données avant la réforme du
      protocole en 2001.
    \item nous sommes à la recherche des données brut de 1989-2000
    \end{itemize}
  \item \textbf{2001}
    \begin{itemize}
    \item première base de données 
    \item FNat (sous access) très fouillis
    \end{itemize}
     \end{itemize}
\end{frame}


\begin{frame} {Quelques dates clefs}
  \begin{itemize}[<+->]
  \item \textbf{2015}
    \begin{itemize}
    \item révolution : interface en ligne pour la saisie sur un
      miroir des sites faune-france VigiePlume
    \item pb les deux bases de données sont disjointes
    \item réalisation d'un script qui permet de créer la base de
      données unifiée
    \item la base de données est donc créer à partir d'un script qui
      importe les données des deux sources (FNat et VigiePlume)
    \item base de données PostgreSQL et postgis
    \item script disponible pour installation local
    \item et script d'export standardisé (données brut, agrégation au
      points ou au carrés)
    \end{itemize}
   \end{itemize}
\end{frame}

\begin{frame} {Quelques dates clefs}
  \begin{itemize}[<+->]
  \item \textbf{2020}
    \begin{itemize}
    \item c'est re la loose il y a des soucis avec la DB en ligne donc
      pas de mise à jours des tendance cette année
    \end{itemize}
   \end{itemize}
\end{frame}


\begin{frame}{La base de données Postgres}
  \begin{itemize}[<+->]
  \item DB postgis pour jointures spatiales efficaces
  \item Des primary key inéligibles 
    ex : $pk\_inventaire =  pk\_point+date$ \\
    $pk\_point = pk\_carre + num\_point$
   \end{itemize}
\end{frame}


\begin{frame}{La base de données Postgres}
  \begin{itemize}[<+->]
   \item Des tables de données brutes
  \begin{itemize}
  \item $carre$: \textbf{LE carré STOC}
     \newline longitude, latitude, géométrie, altitude, département, commune, $id\_carre$
   \item $point$: \textbf{un carré}
     \newline longitude, latitude, géométrie, altitude, département, commune
   \item $inventaire$: \textbf{un point, un date}
     \newline info passage, conditions envir, ``observateur''
   \item $observation$ : \textbf{un inventaire, une espèce, une classe
       de distance}
     \newline une abondance
   \item $habitat$: \textbf{un inventaire}
     \newline description de l'habitat par l’observateur selon catégories hiérarchiques
   \end{itemize}
    \end{itemize}
  \end{frame}
   


\begin{frame}{La base de données Postgres}
  \begin{itemize}[<+->]
   \item Des tables de données construites
     \begin{itemize}
     \item $point\_annee$: \textbf{un point, une année}
       \newline qualité inventaire, habitat
     \item $carre\_annee$: \textbf{un carré, une année}
       \newline qualité inventaire, habitat
     \end{itemize}
    \end{itemize}
  \end{frame}


\begin{frame}{La base de données Postgres}
  \begin{itemize}[<+->]
   \item Des tables de données génériques
     \begin{itemize}
     \item $species$
       \newline codes, noms, classification
      \item $species\_indicateur\_fonctionnel$
        \newline SSI, STI STrI...
      \item $species\_list\_indicateur$
        \newline agricole, forestier, urbain, généraliste, farmland
        bird, montagne...
      \item codes habitats
         \end{itemize}
    \end{itemize}
  \end{frame}


  \begin{frame}{Une V2 de la base de données Postgres}
    \begin{itemize}[<+->]
    \item couche atlas et birdlife
      \newline stage de Thomas et Elodie
    \item table de traits fonctionnels
    \item veille biblio sur les indicateurs qui sortent
      \newline ex: urbanophilie (Guetté et al. 2017)
    \item table observateurs
    \item amélioration des tables $xxx\_annee$
      \begin{itemize}
      \item description plus fine de habitat
      \item contrôle qualité
        \newline heure de réalisation des points, réalisation de tous les points ...
        les points
      \end{itemize}
    \item table historique des carrés
      \newline pour faire un suivi plus simple
    \end{itemize}
  \end{frame}



%=======================================================
\section{Tendance et variation d'abondance}
% ===============================================



\begin{frame}{Mode de calcul}
  \begin{equation}
    \label{eq:2}
    log(N)\sim annee + carré
  \end{equation}
  \begin{itemize}
  \item année en numérique $\rightarrow$ tendance
  \item année en facteur $\rightarrow$ variation d'abondance
  \end{itemize}
  \begin{itemize}
  \item glm
  \item glmmTMB
  \end{itemize}
\end{frame}


\begin{frame}{Mode de calcul}
  \begin{itemize}
  \item carré
  \item -800 m
  \item que deuxième passage pour migrateur tardif
  \item transiant seulement pour certain gros piaf (rapace)
  \end{itemize}
\end{frame}




\begin{frame}{Mode de calcul}
   \includegraphics[width=\textwidth]{STRTUR_trend_2019_trim}
\end{frame}

\begin{frame}{Mode de calcul}
   \includegraphics[width=0.8\textwidth]{trend_2019_trimvariation_group}
\end{frame}


\begin{frame}{rTRIM}
  \begin{itemize}
  \item outil historique et reconnu à l'échelle européenne TRIM
  \item package rTRIM $\rightarrow$ très très rapide
  \end{itemize}
\end{frame}



\end{document}
