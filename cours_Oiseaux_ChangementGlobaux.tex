\documentclass[10pt]{beamer}
\usepackage[utf8]{inputenc}
%\usepackage[T1]{fontenc}
\usepackage[french]{babel}
% \usepackage{natbib}
%\usepackage{hyperref}
% \usepackage{animate}
%\usepackage{graphicx}
%\usepackage{color}

\pdfminorversion=5 
\pdfcompresslevel=9
\pdfobjcompresslevel=3

%% Theme
% \useinnertheme[shadow=true]{rounded}
% \useoutertheme{shadow}
% \usecolortheme{orchid}
% \usecolortheme{whale}%\usetheme{progressbar}
%\usetheme[]{Berlin}

\useoutertheme[footline=authorinstitutetitle]{miniframes}
\useoutertheme{smoothtree}
\useinnertheme[shadow=true]{rounded}
\usecolortheme{orchid}
\usecolortheme{whale}


\graphicspath{{./images/}} 

% \usecolortheme{progressbar}
% \usefonttheme{progressbar}
% \useinnertheme{progressbar}
% \useoutertheme{progressbar}
%% Logo
% \logo{\includegraphics[height=0.5cm]{./images/Logo.png}}
%% Fond de la page
% \setbeamercolor{background canvas}{bg=couleur}

\title[Les oiseaux face aux changements globaux]{Impacts des changements globaux 
  sur les communautés d'oiseaux\\ \textit{\footnotesize{des populations aux communautés}}}

\author{Romain Lorrillière | \textit{romain.lorrilliere@mnhn.fr}}
\institute{Paris SUD : Master BEE - UE ADAC (Nov 2020)}
\date{19 Novembre 2020}

%\AtBeginSection[]{
%  \begin{frame}
%    \frametitle{\insertsectionhead}
%    \scriptsize \tableofcontents[currentsection,hideothersubsections]
%  \end{frame}
%}


%\AtBeginSubsection[]{
%  \begin{frame}
%    \frametitle{\insertsubsectionhead}
%    \scriptsize \tableofcontents[sectionstyle=show/shaded,subsectionstyle=show/shaded/hide ]
%  \end{frame}
%}
%\addtobeamertemplate{footline}{\insertframenumber/\inserttotalframenumber}

\begin{document}
\maketitle



% \section{Sommaire}
%% =====================
% \begin{frame}
%   \frametitle{Sommaire}
%   \tableofcontents[hideallsubsections]
% \end{frame}




% ===============================================
\section{Pourquoi les oiseaux ?}
% ===============================================

\begin{frame}{Pourquoi les oiseaux ?}
 \includegraphics[width=\textwidth]{chouettes2}
\end{frame}

\begin{frame}[plain]
\begin{center}
	\includegraphics<1>[width=\textwidth]{divOiseaux2_1}
	\includegraphics<2>[width=\textwidth]{divOiseaux2_2}
\end{center}
 \end{frame}



% ----------------------------------------------------------
\subsection{Un groupe ancien} 
% ----------------------------------------------------------

\begin{frame}{Les dinosaures rescapés} 
        \begin{center}
        \includegraphics[width=\textwidth]{the-great-tree-of-life}     
      \end{center}
\end{frame}




\begin{frame}{Les dinosaures rescapés} 
  \begin{columns}
    \begin{column}[c]{0.55\textwidth}
    \begin{small}
    
      \begin{itemize}[<+->]
      \item Groupe ancien
      \item Descendant de petit dinosaures \textit{Theropodes}\\
       \includegraphics<2>[width=.5\textwidth]{Jurassic-Park}   
      \item \textit{Archeopteryx}: branche éteinte \footnotesize{(156 à 150 Ma)}\\
      \includegraphics<3>[width=.5\textwidth]{Archaeopteryx}
      \item Extinction suite à la crise crétacé (volcanisme, météorite), \footnotesize{65,5 Ma env.}
      \end{itemize}
       \end{small}
    \end{column}
    \begin{column}[c]{0.45\textwidth}
      \begin{center}
        \includegraphics<1-5>[width=\textwidth]{bird_evo}     
        \includegraphics<6>[width=\textwidth]{tree_of_piaf}     
      \end{center}
    \end{column}
  \end{columns}
\end{frame}

% ----------------------------------------------------------
\subsection{Avec une forte diversification} 
% ----------------------------------------------------------

\begin{frame}{Diversification : 10000 sp}
  \begin{center}
    \includegraphics[width=0.9\textwidth]{divOiseaux2_2} 
  \end{center}
\end{frame}

\begin{frame}{Diversification}
  \begin{columns}[c]
    \begin{column}[c]{0.35\textwidth}
     \begin{itemize}[<+->]
      \item mode déplacement
      \item stratégie alimentaire
      \item milieu de vie
      \item stratégie reproductive
     \end{itemize}
    \end{column}
    \begin{column}[c]{0.7\textwidth}
      \includegraphics[width=\textwidth]{divOiseaux2_2} \\
    \end{column}
  \end{columns}
\end{frame}

\begin{frame}{Le vol}
  \begin{columns}[c]
    \begin{column}[c]{0.7\textwidth}
      \includegraphics<1->[width=\textwidth]{vol} 
    \end{column}
    \begin{column}[c]{0.3\textwidth}
      \includegraphics<2->[width=\textwidth]{birdman} \\
      \vspace{20pt}
      \includegraphics<3->[width=\textwidth]{plume} 
    \end{column}
  \end{columns}
\end{frame}

% ----------------------------------------------------------
\subsection{La migration} 
% ----------------------------------------------------------

\begin{frame}{La migration}
  \begin{center}
    La migration permet aux oiseaux d’échapper aux conditions rigoureuses de l'hiver qui sévissent sur les sites qu’ils occupent pendant la reproduction\\
    \vspace{20pt}
    \includegraphics[width=.9\textwidth]{volMigration} \\
    \vspace{20pt}
    Phénomène très répandu chez les oiseaux : elle concerne par exemple 40\% des espèces terrestres qui nichent en Europe et en Asie
  \end{center}
\end{frame}

\begin{frame}{La migration}
  \begin{center}
    \includegraphics[width=.9\textwidth]{bird_migration} 
  \end{center}
\end{frame}

\begin{frame}{La migration}
  \begin{center}
  Migrateurs de courte ou longue distance
  \end{center}
  \begin{columns}[c]
    \begin{column}[c]{0.5\textwidth}
      \begin{center}
       \includegraphics<1->[width=.55\textwidth]{migrationCourt} 
      \end{center}
    \end{column}
    \begin{column}[c]{0.5\textwidth}
      \begin{center}
       \includegraphics<2->[width=.55\textwidth]{migrationLong}
      \end{center}
        \end{column}
  \end{columns}
\end{frame}



\begin{frame}{Dispersion}
    \includegraphics[width=0.25\textwidth]{Wandering-albatross-display} 

    \begin{center}
      \includegraphics[width=0.8\textwidth]{wanderingAlbatrosMigration} 
  \end{center}
\end{frame}


% ----------------------------------------------------------
\subsection{Pourquoi les Piafs ?} 
% ----------------------------------------------------------

\begin{frame}{Indicateur de biodiversité}
  \begin{center}
    \includegraphics[width=.8\textwidth]{indicateurs} 
  \end{center}
\tiny{\cite{Balmford2005,Donald2001,Donald2002,Doxa2010,Fisher2009,Gregory2005,Sekercioglu2004,Weibull2003}}
\end{frame}

\begin{frame}{Indicateur international}
  \begin{center}
    \includegraphics[width=.9\textwidth]{oiseauIndicateurEurope} 
  \end{center}
\end{frame}

\begin{frame}{Et surtout...}
  \begin{center}
    \includegraphics[width=.9\textwidth]{AEGCAU} 
  \end{center}
\end{frame}

% =====================================================================
\section{Données et outils}
% =====================================================================




% ----------------------------------------------------------
\subsection{Données opportunistes} 
% ----------------------------------------------------------


\begin{frame}{Les observateurs}
  \begin{center}
    \includegraphics<1>[width=.9\textwidth]{birdwatcher} 
        \includegraphics<2>[width=.9\textwidth]{twitchers} 
            \includegraphics<3>[width=.9\textwidth]{pokemonGo_players} 
              \end{center}

\end{frame}




\begin{frame}{Synthèse d'observation naturaliste}
 \includegraphics[width=.15\textwidth]{visionature_logo}\\
    \begin{center}
        \includegraphics[width=\textwidth]{visioNature4}
  \end{center}
  \footnotesize{source: http://www.faune-iledefrance.org/ Nov. 2015)}
\end{frame}

\begin{frame}{Phénologie: Espèce résidente}
 \begin{columns}
    \begin{column}[c]{0.7\textwidth}
   Mésange bleue \textit{Cyanistes caeruleus}
    \end{column}
    \begin{column}[c]{0.30\textwidth}
     \begin{center}
        \includegraphics[width=\textwidth]{mesangebleuephoto}
  \end{center}
    \end{column}
  \end{columns}
    \begin{center}
        \includegraphics[width=\textwidth]{mesangebleue}
  \end{center}
  \footnotesize{source: http://www.faune-iledefrance.org/ Nov. 2015)}
\end{frame}

\begin{frame}{Phénologie: Visiteuse d'été}
 \begin{columns}
    \begin{column}[c]{0.7\textwidth}
   Hirondelle rustique \textit{Hirundo rustica}
    \end{column}
    \begin{column}[c]{0.30\textwidth}
     \begin{center}
        \includegraphics[width=\textwidth]{hirondellerustiquephoto}
  \end{center}
    \end{column}
  \end{columns}
    \begin{center}
        \includegraphics[width=\textwidth]{hirondellerustique}
  \end{center}
  \footnotesize{source: http://www.faune-iledefrance.org/ Nov. 2015)}
\end{frame}


\begin{frame}{Phénologie: Visiteuse d'hiver}
 \begin{columns}
    \begin{column}[c]{0.7\textwidth}
   Grive mauvis \textit{Turdus iliacus}
    \end{column}
    \begin{column}[c]{0.30\textwidth}
     \begin{center}
        \includegraphics[width=\textwidth]{grivemauvisphoto}
  \end{center}
    \end{column}
  \end{columns}
    \begin{center}
        \includegraphics[width=\textwidth]{grivemauvis}
  \end{center}
  \footnotesize{source: http://www.faune-iledefrance.org/ Nov. 2015)}
\end{frame}



\begin{frame}{Phénologie: Espèce de passage}
 \begin{columns}
    \begin{column}[c]{0.7\textwidth}
   Traquet motteux \textit{Oenanthe oenanthe}
    \end{column}
    \begin{column}[c]{0.30\textwidth}
     \begin{center}
        \includegraphics[width=\textwidth]{traquetmotteuxphoto}
  \end{center}
    \end{column}
  \end{columns}
    \begin{center}
        \includegraphics[width=\textwidth]{traquetmotteux}
  \end{center}
  \footnotesize{source: http://www.faune-iledefrance.org/ Nov. 2015)}
\end{frame}


\begin{frame}{Diversité en 2014}
     \begin{center}
        \includegraphics[width=.75\textwidth]{atlas2014}
  \end{center}
  \footnotesize{source: http://www.faune-iledefrance.org/ Nov. 2015)}
\end{frame}


\begin{frame}{Analyse de données brutes: Hypothèse}
     \begin{block}{Hypothèse forte}<2>
     Effort d'observation constant dans le temps et dans l'espace
     \end{block}
\end{frame}

\begin{frame}{Diversité en 2009}
     \begin{center}
        \includegraphics[width=.75\textwidth]{atlas2009}
  \end{center}
 \footnotesize{source: http://www.faune-iledefrance.org/ Nov. 2015)} 
\end{frame}


\begin{frame}{Synthèse d'observation naturaliste}
 \includegraphics[width=.15\textwidth]{visionature_logo}\\
    \begin{center}
        \includegraphics[width=\textwidth]{visioNature2}
  \end{center}
  \footnotesize{source: http://www.faune-iledefrance.org/ Nov. 2015)}
\end{frame}

% ----------------------------------------------------------
\subsection{Sciences participatives} 
% ----------------------------------------------------------

\begin{frame}{Protocole STOC (Breeding Bird Survey)}
    \begin{center}
      STOC: \textcolor{red}{S}uivi \textcolor{red}{T}emporel des \textcolor{red}{O}iseaux \textcolor{red}{C}ommuns
  \end{center}
 \begin{columns}[c]
    \begin{column}[c]{0.5\textwidth}
      \begin{center}
      2300 carrés suivis au moins une fois depuis 2001 \\
    \includegraphics[width=.8\textwidth]{carreSTOC_map_simple_}
      \end{center}
    \end{column}
    \begin{column}[c]{0.5\textwidth}
    \begin{small}
      \begin{itemize}
      \item  Sites choisie aléatoirement
    \item 10 pts d'écoute de 5mn par site
    \item 2 passages pas an
    \item Description standardisée de l'hab.
    \end{itemize}
    \end{small}
     \begin{center}
       \includegraphics[width=.7\textwidth]{STOC_points}
  \end{center}
    \end{column}
  \end{columns}
 \begin{center}
  175 espèces suivies
  \end{center}
\end{frame}


\begin{frame}{Protocole STOC (Breeding Bird Survey)}
    \begin{center}
        \includegraphics[width=.7\textwidth]{carreSTOC_2017}
  \end{center}
 
\end{frame}


\begin{frame}{STOC: Alouette des champs \textit{Alauda arvensis}}
  \begin{columns}[c]
    \begin{column}[c]{0.5\textwidth}
    \includegraphics[width=.5\textwidth]{Alouette}
     \begin{itemize}[<+->]
      \item  Tendance :
            \begin{itemize}
      \item $-22\%$ depuis 1989, déclin
      \item $-10\%$ depuis 2001, diminution
      \end{itemize}
    \item Distribution géographique par espèce
    \end{itemize}
    \end{column}
    \begin{column}[c]{0.5\textwidth}
     \begin{center}
       \includegraphics<1-3>[width=.9\textwidth]{tendanceAlouette}
    \includegraphics<4->[width=.9\textwidth]{distributionAlouette} 
  \end{center}
    \end{column}
  \end{columns}
\end{frame}


\begin{frame}{STOC: Bruant jaune \textit{Emberiza citrinella}}
  \begin{columns}[c]
    \begin{column}[c]{0.5\textwidth}
    \includegraphics[width=.5\textwidth]{bruantjaune}
     \begin{itemize}
      \item  Tendance :
            \begin{itemize}
        \item $-50\%$ depuis 2001, diminution
      \end{itemize}
    \item Distribution géographique par espèce
    \end{itemize}
    \end{column}
    \begin{column}[c]{0.5\textwidth}
     \begin{center}
       \includegraphics[width=.9\textwidth]{embcit2017}
   \end{center}
    \end{column}
  \end{columns}
\end{frame}




\begin{frame}{STOC: Pigeon ramier \textit{Columba palumbus}}
  \begin{columns}[c]
    \begin{column}[c]{0.5\textwidth}
    \includegraphics[width=.5\textwidth]{pigeonRamier}
     \begin{itemize}[<+->]
      \item  Tendance :
       \begin{itemize}
      \item $+152\%$ depuis 1989, augmentation
      \item $+53\%$ depuis 2001, augmentation
      \end{itemize}
    \item Distribution géographique 
    \end{itemize}
    \end{column}
    \begin{column}[c]{0.5\textwidth}
     \begin{center}
       \includegraphics<1-3>[width=.9\textwidth]{tendancePigeonRamier}
    \includegraphics<4->[width=.9\textwidth]{distributionPigeonRamier} 
  \end{center}
    \end{column}
  \end{columns}
\end{frame}

% ----------------------------------------------------------
\subsection{Le suivis individuel} 
% ----------------------------------------------------------

\begin{frame}{Construction d'un modèle CMR}
  \begin{center}
    \includegraphics<1>[width=\textwidth]{cmr1}   
    \includegraphics<2>[width=\textwidth]{cmr2}
    \includegraphics<3>[width=\textwidth]{cmr3}
    \includegraphics<4>[width=\textwidth]{cmr4}
  \end{center}  
\end{frame}


\begin{frame}{Le STOC capture}
  \begin{center}
    \includegraphics[width=.7\textwidth]{carte_stoc_capture}   
  \end{center}  
\end{frame}


\begin{frame}{Suivie de population:\\ Ex Mouette tridactyle \textit{Rissa tridactila}}
 \begin{columns}
    \begin{column}[c]{0.5\textwidth}
     \begin{center}
       Population suivie en Bretagne \\(Cap Sizun) depuis 1979\\ 
       % mesange 1976 Kergelen 60's
       \includegraphics[width=.6\textwidth]{Mouette-tridactyle_falaise}
  \end{center}
    \end{column}
    \begin{column}[c]{0.5\textwidth}
     \begin{center}
       \includegraphics[width=.4\textwidth]{Mouette-tridactyle_carte}\\
        \includegraphics[width=.7\textwidth]{mouette-tridactyle}
  \end{center}
    \end{column}
  \end{columns}
\end{frame}
\begin{frame}{Suivie de population:\\ Ex Populations d'oiseaux d'austral}
 Populations suivi depuis les années 60 \\
       Kerguelen, Crozet, Terre Adélie...
 \begin{columns}
    \begin{column}[c]{0.5\textwidth}
     \begin{center}
      
       % mesange 1976 Kergelen 60's
       \includegraphics[width=.95\textwidth]{albatros}
  \end{center}
    \end{column}
    \begin{column}[c]{0.5\textwidth}
     \begin{center}
       \includegraphics[width=.95\textwidth]{kerguelen}
  \end{center}
    \end{column}
  \end{columns}
\end{frame}


\begin{frame}{Suivie de population:\\ Ex Mésanges}
 Populations suivis depuis les années 1976 \\
       Corse et zone méditerranéenne métropolitaine (Montpellier) 
 \begin{columns}
    \begin{column}[c]{0.5\textwidth}
     \begin{center}
       \includegraphics[width=.75\textwidth]{mesangebleuemain}
  \end{center}
    \end{column}
    \begin{column}[c]{0.5\textwidth}
     \begin{center}
       \includegraphics[width=.75\textwidth]{mesangenichoir}
  \end{center}
    \end{column}
  \end{columns}
\end{frame}

% ----------------------------------------------------------
\subsection{Migration} 
% ----------------------------------------------------------

\begin{frame}{GPS}
  \begin{center}
    \includegraphics[width=.85\textwidth]{migrMacQueen}   
  \end{center}  
\end{frame}

\begin{frame}{GLS}
  \begin{columns}
    \begin{column}[c]{0.4\textwidth}
     \begin{center}
           \includegraphics[width=\textwidth]{ortolan}
  \end{center}
    \end{column}
    \begin{column}[c]{0.6\textwidth}
     \begin{center}
            \includegraphics[width=\textwidth]{lumiereJour_gls}
  \end{center}
    \end{column}
  \end{columns}
\end{frame}

% ----------------------------------------------------------
\subsection{Modélisation de niche} 
% ----------------------------------------------------------

\begin{frame}{Modélisation de niche actuelle}
  \begin{center}
   \includegraphics[width=\textwidth]{modelNiche1}
  \end{center}
\end{frame}

\begin{frame}{Modélisation de niche futur}
  \begin{center}
   \includegraphics[width=\textwidth]{modelNiche2}
  \end{center}
\end{frame}


\begin{frame}{Modélisation... les consensus (BIOMOD)}
  \begin{center}
   \includegraphics[width=\textwidth]{modelNicheConsensus}
  \end{center}
\end{frame}

\begin{frame}[plain]
  \includegraphics[width=\textwidth]{transition}
\end{frame}





% ===============================================
\section{Changements globaux}
% ===============================================

% ----------------------------------------------------------
\subsection{Les grands sources de contraintes} 
% ----------------------------------------------------------


\begin{frame}{Destruction et dégradation des habitats}
  \includegraphics[width=\textwidth]{destructionHabitat}
\end{frame}

\begin{frame}{La fragmentation en France}
\begin{center}
\includegraphics[width=.7\textwidth]{fragmentationFrance}
 %\includegraphics<1>[width=.5\textwidth]{fragFrance2}
  %\includegraphics<2>[width=\textwidth]{fragmentationFrance}
 \end{center}
\end{frame}

\begin{frame}{La fragmentation en France}
\begin{center}
 \includegraphics[width=.7\textwidth]{routesFrance}
 \end{center}
\end{frame}

\begin{frame}{La fragmentation en France}
\begin{center}
 \includegraphics<1>[width=.7\textwidth]{artif1}
  \includegraphics<2>[width=.7\textwidth]{artif2}
 \end{center}
\end{frame}

\begin{frame}{Déforestation aux États-Unis}
  \includegraphics[width=\textwidth]{deforestationUSA}
\end{frame}
\begin{frame}{L'agriculture}
 \begin{columns}[c]
    \begin{column}[c]{0.4\textwidth}
     \begin{center}
       En Europe: \\ Agriculture $\approx 50\%$  surface 
     \end{center}
    \end{column}
     \begin{column}[c]{0.6\textwidth}
      \begin{center}
     \includegraphics[width=0.9\textwidth]{agriEurope}
      \end{center}
      \begin{tiny}
   \cite{Temme2011}
  \end{tiny}
    \end{column}
  \end{columns}
\end{frame}

\begin{frame}{Evolution paysage agricole}
\begin{center}
Evolution des paysages agricoles au cours des dernières décennies \\
\vspace{10pt}
 \includegraphics[width=\textwidth]{evolutionPaysageAgri}
 \vspace{10pt}
 \textcolor{red}{Homogénéisation du paysage et des pratiques}
 \end{center}
\end{frame}

\begin{frame}{Paradoxe agricole}
\begin{center}
Agriculture : pourquoi s’y intéresser ?\\
\vspace{10pt}
 \includegraphics[width=.6\textwidth]{paradoxeAgri}
 \end{center}
\end{frame}


\begin{frame}{Agriculture et perte de biodiversité}
\begin{center}
 L’agriculture est toujours responsable d’une grande partie de l’érosion de la biodiversité\\
 \vspace{10pt}
\includegraphics[width=.8\textwidth]{agriPerteBiodiv}
 \end{center}
\end{frame}

\begin{frame}{Changements climatiques}
  \begin{columns}
    \begin{column}[c]{0.5\textwidth}
    \begin{center}
      \includegraphics[width=\textwidth]{climat}
    \end{center}
    \end{column}
    \begin{column}[c]{0.5\textwidth}
 \begin{center}
       \includegraphics[width=.5\textwidth]{bears}
      \end{center}
      \begin{itemize}
      \item Temp moy: $+0.74$\degres C  entre 1905 et 2005
      \item Niveau de la mer: +17 cm
      \end{itemize}
      \end{column}
  \end{columns}
  \begin{tiny}
    \begin{itemize}
    \item Rapports du GIEC sur http://www.ipcc.ch
    \item Le climat à découvert - Outils et méthodes en recherche climatique  C. Jeandel et R. Mosseri 
    \end{itemize}
  \end{tiny}
\end{frame}

\begin{frame}{Projections climatiques}
  \includegraphics[width=\textwidth]{scenarioClim} \\
  \vspace{10pt}
  Moyennes multi-modèles et intervalles estimés du réchauffement global en surface 
  \begin{tiny}
    \begin{itemize}
    \item Rapports du GIEC sur http://www.ipcc.ch
    \end{itemize}
  \end{tiny}
\end{frame}


% ----------------------------------------------------------
\subsection{Mode d'actions des contraintes} 
% ----------------------------------------------------------

\begin{frame}{Les contraintes}
  \vspace{5pt}
  \begin{columns}
    \begin{column}[c]{0.5\textwidth}
      \includegraphics<1>[width=0.8\textwidth]{dynamiqueContrainte1}
      \includegraphics<2>[width=0.8\textwidth]{dynamiqueContrainte2}
      \includegraphics<3->[width=0.8\textwidth]{dynamiqueContrainte3}
    \end{column}
    \begin{column}[c]{0.5\textwidth}
      \begin{itemize}
      \item <1-> Contraintes \textit{naturelles}: contraintes liées
        aux systèmes écologiques
        \vspace{20pt}
      \item<3->  Contraintes \textit{anthropiques}: contraintes additionnelles qui ont un effet (+/-) sur la dynamique d'une population
      \end{itemize}
    \end{column}
  \end{columns} 
\end{frame}



% ----------------------------------------------------------
\subsection{Contraintes en interactions} 
% ----------------------------------------------------------

\begin{frame}{Contraintes en interactions}
  \begin{center}
    \includegraphics[width=.8\textwidth]{health_interlinkage_fr} 
  \end{center}
  \begin{tiny}
  OMS (http://www.who.int/globalchange/ecosystems/fr/)
  \end{tiny}
\end{frame}


% =====================================================================
 \section[Les effets sur les populations]{Les populations d'oiseaux face aux changements globaux}
% =====================================================================

\begin{frame} {Les effets sur les populations}
 \includegraphics[width=.8\textwidth]{chouettes2}
\end{frame}

% ----------------------------------------------------------
\subsection{Changement climatiques: phénologie et migration} 
% ----------------------------------------------------------




\begin{frame}{Modification de la phénologie de la migration}
  \includegraphics[width=.9\textwidth]{phenoMigration}
  \\
  \vspace{20pt}
  \begin{center}
    Un timing déterminant pour le succès reproducteur, la survie et la fitness
  \end{center}
\end{frame}

\begin{frame}{Modification de la phénologie de la migration}
 
 \begin{center}
   \includegraphics[width=.6\textwidth]{Gordo_2007_1}

  \end{center}
  \begin{tiny}
\cite{Gordo2007}

  \end{tiny}
\end{frame}

 \begin{frame}{Modification de la phénologie de la migration}
  \begin{center}
   \includegraphics[width=.6\textwidth]{Gordo_2007_2}
  
  \vspace{20pt}
    \textcolor{red}{Quels sont les facteurs à l’origine des décalage des dates d'arrivées?}
    
  \end{center}
  \begin{tiny}
   \cite{Gordo2007}
  \end{tiny}
\end{frame}

\begin{frame}{L'Hirondelle des fenêtres}
  \begin{center}
    \includegraphics[width=.9\textwidth]{hirondelleFenetre}
  \end{center}
  \begin{tiny}
    \cite{Cotton2003}

  \end{tiny}
\end{frame}

\begin{frame}{L'Hirondelle des fenêtres : Date d'arrivées}
  \begin{center}
    \includegraphics[width=.6\textwidth]{hirondelleFenetreAfrique}
  \end{center}
  \vspace{10pt}
  \begin{itemize}
   \item <2> Corrélation de la précocité d’arrivée avec une augmentation des T\degres C en hiver en Afrique sub-saharienne
  \end{itemize}
  \begin{tiny}
   \cite{Cotton2003}

  \end{tiny}
\end{frame}

\begin{frame}{L'Hirondelle des fenêtres : Date de départ}
  \begin{center}
    \includegraphics[width=.6\textwidth]{hirondelleFenetreOxford}
  \end{center}
  \vspace{10pt}
    \begin{itemize}
   \item <2> Corrélation de la précocité de départ avec des fortes chaleurs estivales dans l’Oxfordshire
    \end{itemize}
  \begin{tiny}
    \cite{Cotton2003}
  \end{tiny}
\end{frame}

\begin{frame}{Modification de la phénologie de la migration}
  \begin{center}
    Étude sur 20 espèces migratrices (Oxfordshire, 1971-2000)\\
    \vspace{20pt}
    \includegraphics[width=.8\textwidth]{spMigratrices}
  \end{center}
\begin{tiny}
    \cite{Cotton2003}
  \end{tiny}
\end{frame}

\begin{frame}{Quels gagnants? Quels perdants?}
  \begin{center}
    Migrateurs courte-distance Vs migrateurs longue-distances
    \vspace{10pt}
    \includegraphics[width=.9\textwidth]{bird_migration}
  \end{center}
\end{frame}

\begin{frame}{Quels gagnants? Quels perdants?}
  \begin{center}
    Le cycle annuel
    \vspace{10pt}
    \includegraphics[width=.9\textwidth]{phenoMigrationCourtVsLong}
  \end{center}
\end{frame}

\begin{frame}{Les Pélicans européens}
  
    \begin{center}
        \includegraphics[width=.5\textwidth]{pelican}
    \end{center}
   
\end{frame}

\begin{frame}{Le Pélicans frisé \textit{Pelecanus crispus}}
  
    \begin{center}
         \includegraphics<1>[width=.9\textwidth]{photoPelicanFris}
      \includegraphics<2>[width=.8\textwidth]{pelicanFrise}
  
      \end{center}
   
\end{frame}

\begin{frame}{Le Pélicans blanc \textit{ Pelecanus onocrotalus}}
  
    \begin{center}
         \includegraphics<1>[width=.9\textwidth]{photoPelicanBlanc}
      \includegraphics<2>[width=.8\textwidth]{pelicanBlanc}
  
      \end{center}
   
\end{frame}

\begin{frame}{Les dynamique des populations}
    \begin{center}
      \includegraphics[width=.8\textwidth]{dynPopPelic}
      \end{center}
\begin{tiny}
    \cite{Doxa2012}
  \end{tiny}
\end{frame}

\begin{frame}{Périodes de reproduction}
    \begin{center}
      \includegraphics<1>[width=.8\textwidth]{reproPelicBlanc}
      \includegraphics<2>[width=.8\textwidth]{reproPelicFris}
      \end{center}
\end{frame}

\begin{frame}{Migrateurs courte-distance}
  \begin{center}
    \includegraphics[width=\textwidth]{phenoMigrationCourt}
    \vspace{10pt}
    \begin{itemize}
    \item Restent plus longtemps sur le site de reproduction
    \item Prolongation de la période de reproduction
    \item Des migrateurs de courte distance favorisés par le réchauffement climatique
    \end{itemize}
  \end{center}
\end{frame}

% ----------------------------------------------------------
\subsection{Changement climatiques: phénologie et reproduction}
% ----------------------------------------------------------

\begin{frame}{Avancée de la reproduction}
  \begin{center}
    Exemple de la Mésange charbonnière
    \vspace{10pt}
    \includegraphics[width=.8\textwidth]{phenoReproCharbo}
  \end{center}
  \begin{tiny}
    \cite{Visser2005,Visser2006,Visser1998}
  \end{tiny}
\end{frame}

\begin{frame}{Avancée de la reproduction}
  \begin{columns}[c]
    \begin{column}[c]{0.6\textwidth}
      \begin{center}
        Example de la Mésange charbonnière
        \vspace{10pt}
        \includegraphics[width=.9\textwidth]{phenoReproCharbo2}
      \end{center}
    \end{column}
    \begin{column}[c]{0.4\textwidth}
      \begin{itemize}
      \item Des populations s’adaptent bien (Oxford); d’autres non (Pays-Bas)
      \item Des indices env parfois ``trompeurs'' (photopériode, T\degres C…)
      \item Une plasticité homogène qui permet un bon ajustement global
      \end{itemize}
    \end{column}
  \end{columns}
  \begin{tiny}
    \cite{Charmantier2008}
  \end{tiny}
\end{frame}




\begin{frame}{Avancée de la reproduction}
  \begin{center}  
   Exemple des passereaux nord-américains (Pennsylvanie)
    \vspace{10pt}
    \includegraphics[width=.6\textwidth]{McDermont_etal_2016}
  \end{center}
  \begin{tiny}
    \cite{McDermott2016}

  \end{tiny}
\end{frame}



\begin{frame}{Avancée de la reproduction et habitats anthropisés}
    
  \begin{center}  
    \vspace{10pt}
    \includegraphics[width=.6\textwidth]{Santangeli_et_al_2018}
  \end{center}
  \begin{itemize}[<+->]
   \item  Vanneaux huppé \textit{Vanellus vanellus} et Courlis cendré \textit{Numenius arquata} avancent leur date de reproduction plus vite que les agriculteurs avancent leur date de semie en Finland
\item \textcolor{red}{Les oiseaux s'intallent dans les champs avant les semis au risque de la destruction de leur couvées}
  \end{itemize}

  \begin{tiny}
    \cite{Santangeli2018}

  \end{tiny}
\end{frame}





\begin{frame}{Les chaînes trophiques}
  \begin{center}
    Tendances phénologiques temporelles (1981-2005)
    \vspace{5pt}
    \includegraphics[width=.9\textwidth]{phenoTrophic1}
  \end{center}
  \begin{tiny}
    \cite{Both2009}
  \end{tiny}
\end{frame}

\begin{frame}{Les chaînes trophiques}
  \begin{center}
    Corrélations phénologiques entre niveaux trophiques
    \vspace{5pt}
    \includegraphics[width=.9\textwidth]{phenoTrophic2}
  \end{center}
  \begin{tiny}
    \cite{Both2009}
  \end{tiny}
\end{frame}



% ----------------------------------------------------------
\subsection{Changement climatiques: tolérance thermique et dynamique}
% ----------------------------------------------------------

\begin{frame}{Niche thermique}
  \begin{center}
   \includegraphics[width=.9\textwidth]{rangeThermic}
  \end{center}

\end{frame}

\begin{frame}{Niche thermique exemple}
  \begin{columns}[c]
    \begin{column}[c]{0.5\textwidth}
      \begin{center}
       Bruant jaune \textit{Emberiza citrinella}
        \vspace{10pt}
        \includegraphics[width=.6\textwidth]{bruantjaune}
          \vspace{5pt}
          \includegraphics[width=.8\textwidth]{distributionBruantJaune}   
      \end{center}
    \end{column}
    \begin{column}[c]{0.5\textwidth}
   
     \begin{center}
       Bruant zizi \textit{Emberiza cirlus}
        \vspace{10pt}
        \includegraphics[width=.6\textwidth]{bruantzizi}
          \vspace{5pt}
          \includegraphics[width=.8\textwidth]{distributionBruantZizi}   
      \end{center}
    
    
    \end{column}
  \end{columns}
\end{frame}

\begin{frame}{Réchauffement climatique et tendances à long terme}
  \begin{center}
   \includegraphics[width=.8\textwidth]{tendanceMaxTherm}\\
  \vspace{10pt}
   \begin{itemize}
    \item <2> Corrélation positive entre les tendances sur 25 ans et leur maximum thermique
   \end{itemize}

  \end{center}

\end{frame}


\begin{frame}{Réchauffement climatique et tendances à long terme}
  \begin{columns}[c]
    \begin{column}[c]{0.5\textwidth}
      \begin{center}
   \includegraphics[width=.9\textwidth]{wegge_and_rolstad_2017}\\
 \end{center}
    \end{column}
    \begin{column}[c]{0.5\textwidth}
   
    \begin{itemize}
    \item <2> Le Grand Tétras \textit{Tetrao urrogallus} et le Tétras lyre \textit{Tetrao tetrix} bénéficie du réchauffement climatique en finland
    
   \end{itemize}
    
    \end{column}
  \end{columns}

   \begin{tiny}
    \cite{Wegge2017}
  \end{tiny}

\end{frame}



\begin{frame}{Événements extrêmes : \small{exemple de la canicule de 2003}}
  \begin{center}
   \includegraphics[width=.9\textwidth]{rangeThermEventExtrem}
  \end{center}
  \end{frame}

  
  
\begin{frame}{Réchauffement climatique et tendances à long terme}
 
    \begin{itemize}
    \item  \textcolor{red}{Le réchauffement climatique à globalement un effet positif sur la qualité de la reproduction}
    \item \textcolor{red}{\textbf{Mais attention aux seuils de chaleurs et quand sont les pics. Le printemps 2020 probablement reproduction historiquement mauvaise (données STOC capture)}}
     \item \textcolor{red}{Vrai pour un grand nombre d'espèces}
    
   \end{itemize}
  
\end{frame}
  
% ----------------------------------------------------------
\subsection{Changement climatiques: aire de distribution}
% ----------------------------------------------------------

\begin{frame}{Changements des zones d’hivernage}
  \begin{center}
   Conséquences des hivers plus doux \\
  \vspace{10pt}
  \includegraphics[width=.75\textwidth]{changementZoneHiver}
  \end{center}

\end{frame}

\begin{frame}{Distribution hivernal: Gobemouche à collier}
  \begin{center}
   \includegraphics[width=.75\textwidth]{GobeMoucheCollier}
  \end{center}
\tiny{\cite{Barbet-Massin2009}}
\end{frame}

\begin{frame}{Distribution hivernal: Des dynamiques différentes}
  \begin{center}
   \includegraphics[width=.75\textwidth]{lanioFicedula}
  \end{center}
\tiny{\cite{Barbet-Massin2009}}
\end{frame}

\begin{frame}{Aire de distribution estival}
  \begin{center}
   \includegraphics[width=.75\textwidth]{nicheRepro}
  \end{center}
\tiny{\cite{Barbet-Massin2012}}
\end{frame}


% ----------------------------------------------------------
\subsection{Dégradation des habitats: fragmentation et perturbations}
% ----------------------------------------------------------

\begin{frame}{Indice de Spécialisation des espèces (SSI)}
 \begin{columns}[c]
    \begin{column}[c]{0.55\textwidth}
      \begin{center}
               \includegraphics<1-2>[width=\textwidth]{SSI1}
               \includegraphics<3->[width=\textwidth]{SSI2}
      \end{center}
    \end{column}
    \begin{column}[c]{0.45\textwidth}
      \begin{center}
      Coefficient de variation\\de l'abondance parmi \\les habitats
           \begin{equation}
           SSI_i = \frac{\sqrt{\frac{\sum_{j=1}^K(d_j-\bar{d})^2}{K-1}}}{\bar{d}}
           \end{equation}
     \end{center}
      \begin{itemize}
       \item<2-> Une espèce agricole:\\$SSI_{\textit{Skylark}} = 1.07$
       \item<3>  Une espèce généraliste: \\$SSI_{\textit{Blackbird}} = 0.25$
      \end{itemize}
    \end{column}
  \end{columns}
\end{frame}

\begin{frame}{Les espèces et la fragmentation}
 \begin{columns}[c]
    \begin{column}[c]{0.4\textwidth}
      \begin{center}
     \includegraphics[width=\textwidth]{SSIfragmentation}
      \end{center}
    \end{column}
    \begin{column}[c]{0.6\textwidth}
      \begin{itemize}[<+->]
      \item <2-> Les espèces spécialistes soufrent plus de la fragmentation
      \item <3->Les espèces spécialistes soufrent plus des perturbations
      \end{itemize}
        \begin{tiny}
        \vspace{10pt}
    \cite{Devictor2008a}
  \end{tiny}
    \end{column}
  \end{columns}
\end{frame}

% ----------------------------------------------------------
\subsection{Dégradation des habitats: le milieu marin}
% ----------------------------------------------------------

\begin{frame}{Le milieu marin et la sur-pêche}
 \begin{columns}[c]
    \begin{column}[c]{0.6\textwidth}
      \begin{center}
     \includegraphics[width=.6\textwidth]{CuryEtAl_2011_a}\\
     Fou varié \textit{Sula variegata}
      \end{center}
    \end{column}
      \begin{column}[c]{0.4\textwidth}

      \begin{center}
   
    \includegraphics[width=\textwidth]{CuryEtAl_2011_b}
    \end{center}
    \end{column}
  \end{columns}
\tiny{\cite{Cury2011}}
\end{frame}

\begin{frame}{Le milieu marin et les pollutions}
 \begin{columns}[c]
    \begin{column}[c]{0.5\textwidth}
      \begin{center}
     \includegraphics[width=.65\textwidth]{dechetAlbatros}
          \end{center}
    \end{column}
    \begin{column}[c]{0.5\textwidth}
      \begin{itemize}
      \item déchêts
   
      \end{itemize}
      \vspace{20pt}
        \includegraphics[width=.5\textwidth]{Savoca_etal_2016}
    \end{column}
  \end{columns}
  \begin{tiny}
        \vspace{10pt}
       
        \cite{Cury2011,Savoca2016}
  \end{tiny}

\footnotesize{}
\end{frame}


\begin{frame}{Le milieu marin et les pollutions}
 \begin{columns}[c]
    \begin{column}[c]{0.5\textwidth}
      \begin{center}
  
     \includegraphics[width=\textwidth]{maree_noire}
      \end{center}
    \end{column}
    \begin{column}[c]{0.5\textwidth}
      \begin{itemize}
      \item déchêts
      \item marée noire (Erika env. 150000 oiseaux marin mort)
      \end{itemize}
    \end{column}
  \end{columns}
  \begin{tiny}
      \vspace{10pt}
     
   \cite{Cury2011,Savoca2016}
  \end{tiny}

\end{frame}


% ----------------------------------------------------------
\subsection{Les espèces invasives}
% ----------------------------------------------------------
\begin{frame}{Les espèces invasives}
Un débat complexe scientifique et sociétale  
       \begin{center}
     \includegraphics<1>[width=.8\textwidth]{pyton}
     \includegraphics<2>[width=.8\textwidth]{rat}
     \includegraphics<3>[width=.9\textwidth]{perruche}
	     \includegraphics<4>[width=.9\textwidth]{Bird-and-cat}
      \end{center}
\tiny{\cite{Courchamp2003,Deguines2019,Dove2011,Hernandez-Brito2018,Shiels2014}}
 \end{frame}

\begin{frame}{Le cas du chat}

       \begin{center}
     \includegraphics[width=.5\textwidth]{loss_etal_2013}
   
      \end{center}
       \begin{tiny}
      \vspace{10pt}
      \cite{Loss2013}

  \end{tiny}
 \end{frame}

% ----------------------------------------------------------
\subsection{Le dérangement}
% ----------------------------------------------------------

\begin{frame}{L'effet des courreur}

       \begin{center}
     \includegraphics[width=.8\textwidth]{lethlean_et_al_2017}
   
      \end{center}
       \begin{tiny}
      \vspace{10pt}
        \cite{Lethlean2017}

  \end{tiny}
 \end{frame}

% =====================================================================
 \section[Les effets sur les communautés]{Les communautés d'oiseaux face aux changements globaux}
% =====================================================================


% ----------------------------------------------------------
\subsection{Indicateurs de biodiversité contrasté}
% ----------------------------------------------------------

\begin{frame} {Les effets sur les communautés}
 \includegraphics[width=.8\textwidth]{camargue}
\end{frame}


\begin{frame}{Diminution de la diversité estival}
  \begin{center}
   \includegraphics[width=.9\textwidth]{Huntley_et_al_2006}
  \end{center}
  \begin{tiny}
   \cite{Huntley2006}
  \end{tiny}

\end{frame}



% ----------------------------------------------------------
\subsection{Dégradation des habitats: pratiques agricoles}
% ----------------------------------------------------------



\begin{frame}{Exemple du Pipit farlouse}
 \begin{columns}[c]
    \begin{column}[c]{0.6\textwidth}
     \begin{itemize}
      \item -89\% depuis 1989, déclin
	\item -36\% depuis 2001, diminution
     \end{itemize}
     \begin{small}
\begin{quotation}
     C’est une espèce en fort déclin, qui est à la fois en limite sud d’aire de distribution en France et spécialiste des milieux agricoles. Autant dire que si notre diagnostique est juste, ce déclin devrait malheureusement se poursuivre dans les plaines françaises. Le Pipit farlouse est en déclin également au niveau européen.
 \end{quotation}
    \end{small}
      \begin{tiny}
   http://vigienature.mnhn.fr/page/pipit-farlouse
  \end{tiny}
        \end{column}
     \begin{column}[c]{0.4\textwidth}
      \begin{center}
     \includegraphics[width=0.8\textwidth]{pipitfarlouse}\\
     \includegraphics[width=0.8\textwidth]{pipitFalouse_trend}
      \end{center}
        \end{column}
  \end{columns}
\end{frame}
  
\begin{frame}{Indicateur européen et français}
\begin{center}
 \includegraphics<1>[width=.9\textwidth]{farmlandBird_trend2017}
  \includegraphics<2>[width=.9\textwidth]{trendIndicatorBird}\\
   \end{center}
\end{frame}
  
\begin{frame}{Intensification et communautés}
 \begin{columns}[c]
    \begin{column}[c]{0.45\textwidth}
      \begin{center}
          \includegraphics[width=\textwidth]{agriIntens}
      \end{center}
    \end{column}
    \begin{column}[c]{0.55\textwidth}
     Relation entre le SSI et l'abondance
      \begin{itemize}[<+->]
      \item <2-> Les spécialistes souffrent de l'intensification
      \item <3> Les spécialistes agricoles profitent de l'habitat agricole proche
      \end{itemize}
        \begin{tiny}
   \cite{Filippi-Codaccioni2010}
  \end{tiny}
    \end{column}
  \end{columns}
\end{frame}


\begin{frame}{Les communautés et la fragmentation}
 \begin{columns}[c]
    \begin{column}[c]{0.4\textwidth}
      \begin{center}
     \includegraphics[width=0.9\textwidth]{CSIfragmentation}
      \end{center}
    \end{column}
    \begin{column}[c]{0.6\textwidth}
      \begin{itemize}[<+->]
      \item Les communauté se généralisent face à la fragmentation
      \item Les communautés se généralisent face aux perturbations
      \end{itemize}
       \begin{center}
   \textcolor{red}{Homogénéisation biotique !!!}
      \end{center}
        \begin{tiny}
    \cite{Devictor2008b}
  \end{tiny}
    \end{column}
  \end{columns}
\end{frame}




% ----------------------------------------------------------
\subsection{Dégradation des habitats: fragmentation et perturbations}
% ----------------------------------------------------------

\begin{frame}{Protocole STOC: indicateurs}
  \begin{columns}[c]
    \begin{column}[c]{0.5\textwidth}
     \begin{itemize}[<+->]
       \item  Tendance par espèce :
      Alouette des champs
      \begin{itemize}
      \item 22\% depuis 1989, déclin
      \item 10\% depuis 2001, diminution
      \end{itemize}
       \item Tendance générale des communautés
    \end{itemize}
    \end{column}
    \begin{column}[c]{0.5\textwidth}
     \begin{center}
     \includegraphics<1>[width=.9\textwidth]{tendanceAlouette}
     \includegraphics<4>[width=\textwidth]{tendenceIndicateurOiseau} 
 \end{center}
    \end{column}
  \end{columns}
\end{frame}


\begin{frame}{Protocole STOC: indicateurs}
\begin{center}
  \includegraphics[width=.6\textwidth]{ctri}
 
\end{center}
Approvrissement des régimes alimentaires des communautés
  
\end{frame}





% ----------------------------------------------------------
\subsection{Changement climatiques: aire de distribution}
% ----------------------------------------------------------

\begin{frame}{Suivre ou ne pas suivre sa niche climatique…}
 \begin{center}
   \includegraphics[width=.7\textwidth]{CTI}
 \end{center}

 \begin{tiny}
    \cite{Devictor2008}
  \end{tiny}
\end{frame}

\begin{frame}{Suivre ou ne pas suivre sa niche climatique…}
  \begin{center}
   \includegraphics[width=.6\textwidth]{CTI_lat}
  \end{center}

  \begin{tiny}
   \cite{Devictor2008}
  \end{tiny}
\end{frame}

\begin{frame}{Suivre ou ne pas suivre sa niche climatique…}
  \begin{center}
   \includegraphics[width=.6\textwidth]{CTIevolution}
  \end{center}

  \begin{tiny}
    \cite{Devictor2008}
  \end{tiny}
\end{frame}



\begin{frame}{Suivre ou ne pas suivre sa niche climatique…}
  \begin{center}
         \includegraphics[width=.6\textwidth]{gauzere_etal_2015_2}
  \end{center}

  \begin{tiny}
    \cite{Gauzere2015}
  \end{tiny}
\end{frame}

\begin{frame}{Suivre ou ne pas suivre sa niche climatique…}
  \begin{center}
   \includegraphics[width=.6\textwidth]{gauzere_etal_2016_1}

  \end{center}

  \begin{tiny}
   \cite{Gauzere2016}
  \end{tiny}
\end{frame}



\begin{frame}{Suivre ou ne pas suivre sa niche climatique…}
  \begin{center}

      \includegraphics[width=.4\textwidth]{gauzere_etal_2016_2}
  \end{center}

  \begin{tiny}
   \cite{Gauzere2016}
  \end{tiny}
\end{frame}




% ----------------------------------------------------------
\subsection{Phénomènes beaucoup plus complexe à l'echelle des communautés}
% ----------------------------------------------------------




\begin{frame}{Les indicateurs ne vont pas toujours dans le même sens!}
 \begin{center}
   \includegraphics[width=.32\textwidth]{Monnet_et_al_2014}
 \end{center}

 \begin{tiny}
    \cite{Monnet2014}
  \end{tiny}
\end{frame}


% =====================================================================
 \section{Conclusion}
% =====================================================================


 
\begin{frame}{Caractérisation des contraintes par leur mode d'action}
  \begin{center}
    \begin{columns}
      \begin{column}[c]{0.6\textwidth}
        \includegraphics<1-2>[width=\textwidth]{actionContraintes1}
        \includegraphics<3-4>[width=\textwidth]{actionContraintes2}
        \includegraphics<5-6>[width=\textwidth]{actionContraintes3}
        \includegraphics<7-8>[width=\textwidth]{actionContraintes4}
      \end{column}
      \begin{column}[c]{0.4\textwidth}
        \includegraphics<2>[width=\textwidth]{exemplesContraintes0}
        \includegraphics<4>[width=\textwidth]{exemplesContraintes1}
        \includegraphics<6>[width=\textwidth]{exemplesContraintes2}
        \includegraphics<8>[width=\textwidth]{exemplesContraintes3}
      \end{column}
    \end{columns}
  \end{center}
\begin{tiny}
    \cite{Lorrilliere2012}
  \end{tiny}
\end{frame}


\begin{frame}{Bilan}
\begin{center}
\includegraphics[width=.8\textwidth]{global_change_bird_pautasso2012}
\end{center}
\tiny{\cite{Pautasso2012}}
\end{frame}


\begin{frame}{Pas que les oiseaux}
\begin{center}
\includegraphics[width=.6\textwidth]{Ripple_et_al_2017}
\end{center}
\begin{tiny}
 \cite{Ripple2017a}
\end{tiny}

\end{frame}



\begin{frame}[plain]
\begin{center}
\begin{huge}
 Merci de votre attention...
\end{huge}
\includegraphics[width=\textwidth]{guepier}
\end{center}
 \end{frame}



\begin{frame}[allowframebreaks]
\begin{footnotesize}
        \frametitle{Réferences}
        \bibliographystyle{apalike}
        \bibliography{bib_files/biblio}
\end{footnotesize}
\end{frame}

\end{document}
